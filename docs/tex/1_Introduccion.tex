\capitulo{1}{Introducción}

La Universidad de Burgos (UBU) cuenta con más de 1000 equipos informáticos distribuidos en aulas entre 8 y 10 centros.
Cada una de estas aulas dispone de dispositivos destinados a satisfacer una serie de necesidades distintas, como por ejemplo,
la realización de operaciones matemáticas complejas, redacción de documentos, edición de contenido multimedia, entre otras.

Sin embargo, un mismo equipo no está destinado a un único uso y con las aplicaciones instaladas pueden permitir el acceso a otros
recursos y dispositivos no deseados, esto supone un riesgo para la seguridad de la información y puede derivar en un uso
indebido, como acceder a información no autorizada durante una prueba de evaluación.

En los últimos años, se ha observado un incremento de las peticiones entre los docentes, para limitar los recursos a
los que se pueden acceder desde los ordenadores.
Por este motivo, surge la necesidad de implementar una gestión centralizada en las aulas, que permita, de una manera sencilla, controlar
los accesos de los distintos equipos, atendiendo a características diferentes, como nombres de equipo, aulas, centros\ldots

De acuerdo con Microsoft\cite{microsoftQuAccess}existen cuatro formas de control de acceso:
    \begin{itemize}
        \item Control de accesso dicrecional (DAC): un propietario concede acceso a otros usuarios.
        \item Control de acceso basado en roles (RBAC): se otorga acceso según funciones empresariales definidas.
        \item Control de acceso obligatorio (MAC): se concede acceso en forma de autorización regulado por una autoridad central.
        \item Control de acceso basado en atributos (ABAC): se da de manera flexible en función de variables y condiciones.
    \end{itemize}

Actualmente, los usuarios de los equipos, mediante el protocolo 802.1X\cite{eswiki:802.1X} han de ingresar sus credenciales
y tras una autentificación satisfactoria, se les asigna una dirección IP, que les permite hacer uso del dispositivo, navegar
por Internet, acceder a los distintos recursos de red\ldots.

Existen herramientas que pueden ayudar a limitar el acceso a los recursos, como software específico, no obstante,
estad opciones pueden resultar costosas.
También se pueden configurar uno a uno los permisos de cada dispositivo, pero ello requiere
que manualmente, un administrador tenga que cambiar los distintos permisos, y llevar un control de cada uno de ellos.

La plataforma propuesta, busca optimizar costes y ofrecer una solución práctica, diseñada a medida para la UBU.
A diferencia de otras propuestas, esta iniciativa se enfoca en resolver problemas reales sin añadir funciones innecesarias
que acaban dificultando el uso.
Para su implementación, solo requiere un servidor web encargado de gestionar las configuraciones y una puesta a punto inicial
en cada equipo para su registro en el sistema.
Posteriormente, un software se encargará de aplicar las configuraciones
necesarias en cada dispositivo de forma automática.
