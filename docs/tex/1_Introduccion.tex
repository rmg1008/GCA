\capitulo{1}{Introducción}

La Universidad de Burgos (UBU) cuenta con más de 1000 equipos informáticos distribuidos en las aulas de sus centros;
cada una de ellas dispone de dispositivos destinados a satisfacer necesidades muy diversas (resolución de operaciones
matemáticas complejas, redacción, edición multimedia\ldots etc.).
No obstante, dado que un mismo equipo no está enfocado a un único uso, el acceso ilimitado a todo tipo de herramientas
de las que dispone puede derivar en un uso indebido por parte del usuario y suponer un riesgo para la seguridad de la
información: en este contexto, por ejemplo, puede utilizarse fraudulentamente por los estudiantes a la hora de
realizar una prueba de evaluación, como podría ser el acceso a información no autorizada o transferir archivos entre
dispositivos de red.

En los últimos años, se ha observado un incremento de las peticiones entre los docentes para limitar los recursos a
los que se pueden acceder desde los ordenadores.
Por este motivo, surge la necesidad de implementar una gestión centralizada en las aulas, que permita de una manera
sencilla controlar los accesos de los equipos, atendiendo a características diferentes, como nombres de equipo, aulas,
centros\ldots

Actualmente disponemos de herramientas comerciales capaces de gestionar el acceso a los recursos de manera eficiente;
sin embargo, su elevado coste económico puede hacerlas poco rentables.
Por otro lado, optar por un enfoque manual —donde un
administrador configura y audita cada dispositivo individualmente— conlleva, inevitablemente, un notable aumento de
la carga de trabajo y, en consecuencia, la posible necesidad de incorporar más personal, además de elevar el riesgo
de errores humanos.

La solución que se propone en este proyecto es crear una plataforma para la Gestión Centralizada de las Aulas (GCA),
con la que se pretende optimizar costes y ofrecer una solución práctica, diseñada a medida para la UBU\@.
El gran potencial de esta iniciativa es que se enfoca en resolver problemas reales sin añadir funciones
innecesarias que acaban dificultando su uso.

GCA es capaz de:
\begin{enumerate}
    \item \textbf{Registrar} cada dispositivo en un servidor centralizado utilizando una aplicación cliente.
    \item \textbf{Aplicar} de forma automática políticas de control de acceso (IP, puertos, servicios\ldots)
    \item \textbf{Gestionar} remotamente las configuraciones individualmente o por grupos.
\end{enumerate}

En los apartados siguientes del presente proyecto se expondrán de forma más amplia y pormenorizada los objetivos
específicos de este, así como una guía del trabajo de programación llevado a cabo bajo una metodología definida.

\imagen{logo.png}{Logo de GCA}{0.7}
