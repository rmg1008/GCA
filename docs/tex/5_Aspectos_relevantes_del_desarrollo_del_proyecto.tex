\capitulo{5}{Aspectos relevantes del desarrollo del proyecto}

En este punto se detallan los aspectos más relevantes en el desarrollo del proyecto, desde la primera reunión con el tutor para la
elección del tema, hasta la entrega del mismo.

\section{Elección del proyecto}\label{sec:eleccion-del-proyecto}

En la primera reunión con el tutor, se enumeraron los posibles temas para el Trabajo de Fin de Grado.

El motivo de la elección de este tema viene motivado por experiencias de desarrollo pasadas, donde realicé una aplicación
web para la gestión de las escuelas deportivas del centro donde cursé el CFGS de Desarrollo de Aplicaciones Web (DAW).

\section{Métodos de trabajo}\label{sec:metodos-de-trabajo}

A pesar de no contar con un equipo de trabajo, se decidió seguir una metodología ágil, concretamente Scrum; al no
contar, evidentemente, con un equipo de trabajo, no se ha aplicado de una manera estricta, pero se han seguido los
principios básicos de la misma:
\begin{itemize}
    \item \textbf{Sprint}: se han realizado sprints de dos semanas, donde se planifica el trabajo a realizar
    según un orden de prioridades.
    \item \textbf{Reuniones semanales}: cada semana, los lunes concretamente, se mantuvieron reuniones con el tutor
    para revisar el avance del proyecto (a modo de \textit{demo}) y resolver dudas.
    \item \textbf{Backlog}: en esta lista se han ido añadiendo, durante el curso del desarrollo del proyecto,
    todo tipo de tareas a realizar (de análisis, documentación,
    diseño, desarrollo, pruebas\ldots).
\end{itemize}

En cuanto al desarrollo con Scrum, no se define un conjunto de fases específicas, sino que se trabaja de una forma
iterativa e incremental: en cada sprint se van añadiendo nuevas funcionalidades y se van corrigiendo errores.

\section{Fase de análisis}\label{sec:fase-de-analisis}

Durante este periodo, se dedicaron varias sesiones a la recopilación de información sobre el tema a desarrollar, las distintas
necesidades de los usuarios y la manera de satisfacerlas.

La idea principal era crear una herramienta que permitiese gestionar, de forma centralizada, el acceso a la red de
los dispositivos que dispone la Universidad.
Para ello, se estudió la forma de gestionar este acceso desde la consola de comandos y se descubrieron herramientas que
permitían esta gestión, como el firewall de Windows y Netsh.

Uno de los principales problemas que surgieron fue la manera de identificar a cada dispositivo de forma única.
La primera idea fue utilizar el GUID del dispositivo, pero se descubrió que los dispositivos del centro son
clonados, y, por lo tanto, comparten el mismo, por lo que se tuvo que descartar irremediablemente esta vía.

Ante esta dificultad, se planteó la posibilidad de combinar este identificador GUID con el nombre del dispositivo para
crear una huella única.

Estos dispositivos se guardarían en una base de datos junto con las configuraciones que se pretendan aplicar.
Un detalle importante que se especificó en esta fase, es que contaríamos con grupos o localizaciones de dispositivos
que formarían una jerarquía y, así, se podrían aplicar configuraciones de carácter general a todos los dispositivos que
pertenecen a cada grupo.

\section{Fase de diseño}\label{sec:fase-de-diseno}

Barajamos la posibilidad de utilizar un script que se mantuviese en ejecución en segundo plano, pero se
descartó por ser poco eficiente y seguro.
En su lugar, decidimos utilizar un ejecutable y mantenerlo activo de forma continua, mediante el programador de
tareas de Windows; de esta manera, se restringe la posibilidad de que un usuario pueda modificar el script y ejecutar
comandos que puedan comprometer la seguridad del sistema.
Las tecnologías propuestas para el ejecutable fueron Java y Python, pero la creación de un ejecutable (.exe) en Java
causó muchos problemas y no se pudo realizar.
Con un archivo JAR sí se pudo crear, pero generaba la necesidad de tener instalado Java en cada dispositivo.
Por eso se optó por utilizar Python, ya que permite crear un ejecutable independiente del sistema operativo.

En cuanto a la base de datos (BD) se decidió utilizar un modelo relacional, porque los datos a almacenar son de
carácter estructurado y van a tener relación entre ellos.

Para la API se decidió utilizar un modelo REST, ya que es el más utilizado en la actualidad y permite una fácil
interacción entre el cliente y el servidor.
La herramienta elegida fue Spring Boot y Java, debido a que está enfocada a la creación de servicios REST y cuenta con una
gran cantidad de herramientas y librerías ya instaladas.

Por último, se decidió crear una interfaz web para realizar toda la gestión de manera centralizada: por compatibilidad
y por ser uno de los frameworks actuales más utilizados, se decidió utilizar Angular.

Con toda esta estructura, aseguramos que la aplicación sea escalable y fácil de mantener, e incluso si se quisiese
crear una aplicación para smartphone en un futuro, se podría reutilizar la parte backend, agilizando el proceso de
desarrollo de la app.

\section{Fase de desarrollo}\label{sec:fase-de-desarrollo}

La construcción del código comienza con el ejecutable en Python: una primera versión con una ventana
y un formulario para iniciar sesión, y, tras la autenticación, un botón para registrar el dispositivo
en la base de datos.

El principal desafío consistió en crear un instalador.
Para ello, se utilizó Inno Setup combinado con el ejecutable (.exe) generado mediante la librería PyInstaller; esto
permitió no solo instalar la aplicación en el dispositivo, sino también mantenerla en ejecución en segundo plano,
mostrando un icono en la barra de tareas.


El siguiente paso fue la creación de la API con la base de datos, que se realizó de forma paralela al ejecutable.
La configuración inicial de la API con Spring Boot fue lo más tedioso, porque fue necesario instalar todas aquellas
dependencias que se van a utilizar, además de establecer la estructura de carpetas y archivos.
La forma de autenticación elegida fue mediante JWT, una de las más utilizadas en la actualidad, lo que hace posible
una fácil interacción entre el cliente y el servidor.
Se implementaron, de manera gradual, los distintos casos de uso de la API, generando diferentes endpoints para
cada uno de estos.
Añadimos SonarCloud para el análisis de la calidad del código con GitHub Actions; sin embargo, inicialmente, se
desactivaron los tests porque la BD no es en memoria ni se contaba con Docker, además tuvimos que utilizar una versión
de Java inferior a la 21, ya que SonarCloud soporta hasta la 17.

Para el desarrollo de la interfaz web, fue necesario descargar Node.js y Angular CLI con tal de generar el proyecto
y sus dependencias.
A partir de este paso, cualquier librería externa se añade mediante el gestor de paquetes NPM\@.
Utilizamos Tailwind CSS + DaisyUI para el diseño de la interfaz, ya que incluye estilos predefinidos para
maquetar la aplicación.

La interfaz web se desarrolla de forma paralela a la API: primero se escriben los tests de la API, y a continuación, se escribe
el código necesario para que pasen, aplicando el desarrollo dirigido por pruebas (TDD); por último, se implementa la interfaz web.
Una de las funcionalidades más difíciles de implementar fue la estructura arbórea de los grupos situada en el menú
izquierdo de la interfaz; tenía que cargar de forma dinámica el listado de dispositivos que pertenecen a cada uno en
la parte derecha, además de mantener la concordancia con los datos.
Esto es clave, porque facilita al usuario la gestión de los dispositivos de una manera intuitiva y rápida.

La gestión de plantillas y comandos se implementa de una manera en la que se crean los comandos ``fijos``, y estos
son añadidos a las distintas plantillas.
No obstante, se descubrió que esta lógica no era muy eficaz.
Si queríamos, por ejemplo, añadir un valor a un comando de una dirección IP, teníamos que crear un comando para cada
dirección, lo que resultaría tremendamente tedioso y se perdería la agilidad que pretendemos conseguir de forma general con GCA.
Para solventar este problema, se estableció una estructura básica para añadir valores tal que el valor de un comando que contenga \verb|{{variable}}|
sirve para cambiar el valor de esa variable al asignarse a una plantilla.

Cuando terminamos con la construcción de la API (o, al menos, el desarrollo de aquellas estructuras que se consideraron
necesarias para su funcionamiento), se decidió mejorar la aplicación Python para el cliente.
Inicialmente, había demasiada lógica en cada archivo .py, por lo que creamos clases más pequeñas con una lógica bien definida.
El funcionamiento de mostrar / ocultar botones limita el desarrollo y añade lógica innecesaria, por lo que se opta por
crear frames o páginas y añadir en estas los elementos necesarios.
En nuestro caso, creamos una página para iniciar sesión y otra para mostrar el panel de usuario una vez iniciada.
Otro punto a destacar aquí es que, cuando intentamos realizar el registro de un equipo y aplicar la configuración,
no se sabía qué estaba ocurriendo, más allá de alguna alerta programada.
Introdujimos un recuadro para ir mostrando todos los logs de la aplicación mediante diferentes colores: los de
información en azul, advertencias en naranja y errores en rojo.

Tras conseguir una primera aplicación de la configuración asignada a un dispositivo, comenzaron los problemas:
\begin{itemize}
    \item Al deshabilitar la red para el dispositivo, se perdía la conectividad a la API.
    \item Tras aplicar la configuración de plantilla, quitar la asignación y añadir otra diferente, no se aplicaba
    correctamente.
    \item Al bloquear toda conexión saliente de la interfaz, no podíamos conectar de nuevo a la API.
\end{itemize}

Tuvimos que trabajar en el algoritmo para aplicar la configuración; así, el funcionamiento queda definido de la siguiente
manera:

\begin{enumerate}
    \item Se restaura la copia de seguridad.
    \item Se eliminan los registros de plantillas anteriores.
    \item A partir de aquí, se comprueba de manera periódica la existencia de configuración a aplicar.
    \item Se obtiene el valor de la huella digital almacenada en el dispositivo.
    \item Si existe una huella digital, se solicita al servidor la configuración a aplicar.
    \item Si se reciben datos de configuración:
    \begin{itemize}
        \item Se comprueba si la configuración obtenida es más actual que la que tiene el equipo.
        \item Si hay cambios, se restaura la copia de seguridad y se aplican los comandos.
        \item Se actualizan los valores en el registro del dispositivo.
    \end{itemize}
    \item Si no hay configuración o está vacía, se revisa si se ha eliminado la asignación del equipo y restaura.
    \item Se espera el tiempo definido antes de reiniciar el ciclo.
\end{enumerate}

La copia de seguridad se realiza una única vez en el momento de registrar el equipo, de forma que nos aseguramos de que
el dispositivo pueda volver a su estado original ante cualquier imprevisto.
Esta copia incluye la configuración del firewall y la configuración de red.
En el proceso de restauración también se habilitan de nuevo las interfaces de red y se configuran las reglas de firewall
para mantener la conexión a las URL's configuradas en el json como permitidas.

\section{Fase de pruebas}\label{sec:fase-de-pruebas}

Las pruebas se realizan de manera continua a lo largo de todo el ciclo del desarrollo: para el backend utilizamos las
librerías JUnit para pruebas unitarias, y Spring Boot Test para pruebas de integración.

En primer lugar, escogimos la misma BD con la que inicia la API para ejecutar los tests, pero esto generaba
conflictos con los datos, por lo que se decidió utilizar una BD en memoria, H2, para realizar las pruebas.
Para la configuración de la BD en memoria, creamos un nuevo archivo de application.properties dentro del directorio
de test/resources.
Este archivo contiene la configuración de la BD en memoria, y se carga automáticamente al ejecutar los tests.
Aquí surgió un problema: los scripts se ejecutaban antes de que Hibernate crease la BD dando lugar a errores de inserción.
Para que funcionase correctamente, según lo propuesto en StackOverflow\cite{stackoverflowSpringBootError},
fue necesario añadir la siguiente propiedad spring.jpa.defer-datasource-initialization=true para invertir el orden, y
que primero se generase la estructura y después se insertase los datos.

Una vez configurada la BD en memoria, se vuelven a activar los tests en SonarQube Cloud, pero se producen una serie
de errores relacionados con el contexto de Spring.
Aquellos test que hacen uso de la anotación @SpringBootTest levantan el contexto, permiten inyectar beans y utilizar
todos los componentes, y como resultado, se ejecutaban de nuevo los scripts de la base de datos.
Existía una confusión entre las tablas generadas por JPA y los scripts que se ejecutaban para la BD en memoria,
por lo que se ajustó poniendo todos los nombres de tablas en minúsculas, y para los tests de spring,
estableciendo el perfil activo a la hora de ejecutar los tests de ''tests''.
De esta manera, solamente se ejecutan aquellas propiedades dentro de la carpeta test/.

Mediante este enfoque conseguimos un entorno de pruebas aislado, reproducible y compatible con SonarQube.

Las pruebas de la aplicación cliente se hicieron de manera manual, tratando de cubrir todos aquellos casos que
se puedan dar, como por ejemplo:
\begin{itemize}
    \item Asignar una plantilla de configuración vacía.
    \item Asignar una plantilla a un dispositivo y al grupo directo al que pertenece y ver qué plantilla se obtiene.
    \item Eliminar la plantilla asignada y ver qué sucede con la configuración.
    \item Reiniciar / Apagar el dispositivo y comprobar que se aplique la configuración correcta.
\end{itemize}

De esta manera, aseguramos que la funcionalidad más importante esté cubierta.

Por último, para la aplicación web, al alojar un número reducido de funcionalidades, simplemente se comprueba que
toda la gestión de dispositivos, grupos, comandos y plantillas sea correcta, es decir, que se pueda añadir, eliminar,
modificar y listar.