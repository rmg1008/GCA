\capitulo{2}{Objetivos del proyecto}

Existen diferentes objetivos que han derivado en el nacimiento del proyecto:

\section{Objetivos generales}\label{sec:objetivos-generales}
    \begin{itemize}
        \item Crear una plataforma para el control de acceso a los recursos de los dispositivos del centro.
        \item Contribuir a la automatización de procesos de la Universidad.
        \item Solucionar las necesidades del personal docente.
        \item Guardar las distintas configuraciones y permitir una gestión eficiente.
    \end{itemize}


\section{Objetivos técnicos}\label{sec:objetivos-tecnicos}
    \begin{itemize}
        \item Crear una aplicación web para permitir la gestión de los dispositivos de una manera sencilla e intuitiva.
        \item Desarrollar una aplicación Java para gestionar las configuraciones a realizar.
        \item Utilizar herramientas de control de versiones como Git y GitHub.
        \item Utilizar una arquitectura MVC (Modelo-Vista-Controlador).
        \item Aplicar una metodología Ágil (Scrum) en el desarrollo del proyecto.
            \begin{itemize}
                \item Utilizar ZenHub para gestión de proyectos.
            \end{itemize}
        \item Aplicar TDD (Test Driven Development) durante todo el desarrollo, para garantizar calidad en el producto.
    \end{itemize}


\section{Objetivos personales}\label{sec:objetivos-personales}
    \begin{itemize}
        \item Aportar una herramienta beneficiosa a la Universidad.
        \item Aplicar los conocimientos aprendidos durante la ingeniería.
        \item Ampliar conocimientos técnicos:
        \begin{itemize}
                  \item Gestión de dispositivos.
                  \item Java.
                  \item Angular.
                  \item Herramientas de testeo.
        \end{itemize}
    \end{itemize}