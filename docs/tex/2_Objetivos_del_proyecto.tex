\capitulo{2}{Objetivos del proyecto}

Existen diferentes objetivos que han motivado el nacimiento del proyecto:

\section{Objetivos generales}\label{sec:objetivos-generales}
    \begin{itemize}
        \item Crear una plataforma para el control de acceso a los recursos de los dispositivos del centro.
        \item Contribuir a la automatización de procesos de la Universidad.
        \item Solucionar las necesidades del personal docente.
        \item Guardar las distintas configuraciones y permitir una gestión eficiente.
        \item Permitir ahorrar en costes al desarrollar una aplicación a medida.
    \end{itemize}


\section{Objetivos técnicos}\label{sec:objetivos-tecnicos}
    \begin{itemize}
        \item Crear una aplicación web para permitir la gestión de los dispositivos de una manera sencilla e intuitiva.
        \item Desarrollar una aplicación Java para gestionar las configuraciones a implementar.
        \item Utilizar herramientas de control de versiones como Git y GitHub.
        \item Utilizar una arquitectura MVC (Modelo-Vista-Controlador).
        \item Aplicar una metodología Ágil (Scrum) en el desarrollo del proyecto.
        \item Utilizar Zube para gestión de proyectos.
        \item Aplicar TDD (Test Driven Development) durante todo el desarrollo, para garantizar calidad en el producto.
        \item Implementar un marco DevOps orientado a mejorar la colaboración entre desarrollo y operaciones.
    \end{itemize}

\section{Objetivos personales}\label{sec:objetivos-personales}
    \begin{itemize}
        \item Aportar una herramienta práctica, útil y beneficiosa para la Universidad.
        \item Aplicar los conocimientos aprendidos durante el Grado.
        \item Ampliar conocimientos técnicos:
        \begin{itemize}
                  \item Gestión de dispositivos.
                  \item Python
                  \item Java.
                  \item Angular.
                  \item Herramientas de testeo.
        \end{itemize}
        \item Profundizar en la gestión de políticas de control de acceso mediante comandos.
        \item Aprender a crear aplicaciones de escritorio.
    \end{itemize}