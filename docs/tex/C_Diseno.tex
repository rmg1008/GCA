\apendice{Especificación de diseño}

\section{Introducción}

Esta sección describe el diseño del sistema, incluyendo el diseño de datos, diseño arquitectónico, diseño procedimental
y diseño de interfaces.
El objetivo es proporcionar una visión general de cómo se estructuran los componentes del sistema y cómo interactúan entre sí
para comprender mejor su funcionamiento.

\section{Diseño de datos}

GCA trabaja con los siguientes tipos de datos:
\begin{itemize}
    \item \textbf{Usuarios}: figuras que interactúan con el sistema mediante un rol específico.
    \item \textbf{Grupos}: almacenan dispositivos y otros grupos, permitiendo una organización jerárquica, además de
    almacenar la plantilla asignada.
    \item \textbf{Dispositivos}: guardan información sobre cada dispositivo, el grupo al que pertenecen, así como
    su sistema operativo y la plantilla asignada.
    \item \textbf{Comandos}: nombre, descripción y valor del comando que se ejecutará en los dispositivos.
    \item \textbf{Plantillas}: nombre, descripción y lista de comandos que se ejecutarán en los dispositivos mediante
    un orden de ejecución.
    \item \textbf{Roles}: los roles que pueden tener los usuarios, como administrador o gestor.
    \item \textbf{Sistemas Operativos}: el conjunto de sistemas operativos que pueden tener los dispositivos.
\end{itemize}

\imagenAncho{ER.pdf}{Diagrama entidad relación}{1}

\imagenAncho{diagram.png}{Diagrama exportado en IntelliJ}{1}

\section{Diseño arquitectónico}

Antes de comenzar con la explicación de la arquitectura, hay que distinguir entre las tres partes en las que se divide este proyecto:
\begin{itemize}
    \item \textbf{Backend}: aplicación desarrollada con Java y Spring Boot.
    Se trata de una API la cual consumen el resto de componentes.
    \item \textbf{Frontend}: aplicación Angular utilizada para llevar a cabo la gestión de dispositivos.
    \item \textbf{Cliente}: aplicación Python que genera un instalador para ejecutar en los distintos dispositivos
    sobre los que se quieren aplicar las configuraciones.
\end{itemize}

\subsection{Backend}


El patrón MVC (Modelo-Vista-Controlador) es un patrón de diseño que separa la lógica de negocio, la interfaz de usuario
y el control de flujo de la aplicación\cite{springMvc}.

En el caso del modelo, este se obtiene con ayuda de JPA, que mapea las entidades de la base de datos a objetos Java que se
pueden manipular, sin necesidad de escribir consultas SQL directamente.
La interacción con la base de datos se consigue mediante el uso del patrón de repositorio, que permite realizar
operaciones CRUD (Crear, Leer, Actualizar, Eliminar) sobre las entidades de forma sencilla.

La vista se mantiene totalmente independiente del backend y se gestiona desde el frontend, que explicaremos en la
siguiente sección.

El controlador, por su parte, se encarga de recibir las peticiones HTTP y dirigirlas a los servicios correspondientes.
Para los servicios se ha aplicado una arquitectura de microservicios, donde cada servicio es responsable de una
parte específica de la lógica de negocio: uno para gestionar usuarios, otro para grupos, dispositivos\ldots
Se consigue así una mayor modularidad y escalabilidad, permitiendo que cada servicio pueda evolucionar de forma
independiente.

El backend se estructura de la siguiente manera:
\imagen{diagrama_paquetes.png}{Diagrama de paquetes del backend}{1}

Los paquetes que lo componen son:
\begin{itemize}
    \item \textbf{Config}: se incluyen archivos para la configuración de la aplicación.
    \item \textbf{Controller}: contiene los controladores que gestionan las peticiones HTTP y utilizan los servicios.
    \item \textbf{Domain}: contienen las clases equivalentes a cada una de las tablas de la base de datos.
    \item \textbf{Dto}: se utilizan para enviar datos de los modelos a la vista, evitando así enviar información
    sensible o innecesaria.
    \item \textbf{Exception}: contiene las excepciones personalizadas que pueden lanzar el sistema.
    \item \textbf{Repository}: contiene las interfaces que extienden de JPA y permiten realizar operaciones CRUD.
    \item \textbf{Security}: contiene la configuración de seguridad de la aplicación; incluye la autenticación y
    autorización.
    \item \textbf{Service}: implementa la lógica de negocio y utiliza los repositorios.
    \item \textbf{Util}: clases de utilidad que pueden ser utilizadas en diferentes partes del sistema.
\end{itemize}

La estructura completa de clases del backend es la siguiente:

\imagenAncho{clases.png}{Diagrama de clases del backend}{0.8}

\subsection{Frontend}

El front se estructura de forma similar al backend, siguiendo el patrón MVC, pero adaptado a Angular.

Dentro de la carpeta \texttt{src/app} se encuentra dividida la estructura de la siguiente manera:
\begin{itemize}
    \item \textbf{Components}: contiene los componentes de la aplicación, que son las partes reutilizables de la
    interfaz de usuario.
    \item \textbf{Model}: contiene las clases que representan los modelos de datos utilizados en la aplicación.
    \item \textbf{Services}: contiene los servicios que se encargan de realizar peticiones HTTP al backend y gestionar
    la lógica de negocio del frontend.
\end{itemize}

\imagenAncho{angular.png}{Diagrama de componentes Angular}{1}

\subsection{Cliente}

El cliente Python se estructura de la siguiente manera:

\imagen{python_diagram.png}{Paquetes de Python}{1}

Se divide en los siguientes paquetes:
\begin{itemize}
    \item \textbf{libs}: contiene las librerías utilizadas.
    \item \textbf{libs.core}: contiene las clases que implementan la lógica principal.
    \item \textbf{libs.pages}: contiene las páginas que se utilizan para interactuar con el usuario.
\end{itemize}

El punto de entrada del cliente es el archivo \texttt{main.py}, que se encarga de inicializar la aplicación y
cargar la configuración necesaria.

\section{Diseño procedimental}

Esta sección muestra los diagramas de secuencia de los principales procesos del sistema, los cuales describen cómo interactúan
los diferentes componentes y cómo se llevan a cabo las operaciones:

\imagenAncho{loginSequence.png}{Diagrama de secuencia del inicio de sesión.}{1}

\imagenAncho{templateSequence.png}{Diagrama de secuencia para la obtención de plantillas.}{1}

\imagen{algorithmSequence.png}{Diagrama de secuencia del algoritmo principal.}{0.6}

\section{Diseño de interfaces}

Se han creado prototipos de las principales pantallas del sistema para facilitar la comprensión de la interfaz y partir
de una base visual para el desarrollo.

\imagenAncho{login}{Prototipo inicio de sesión.}{1.0}

\imagenAncho{grupos}{Prototipo listar grupos y dispositivos.}{1.0}

\imagenAncho{menuGrupos}{Prototipo acciones grupos.}{1.0}

\imagenAncho{modificarDisp}{Prototipo modificar dispositivo.}{1.0}

\imagenAncho{crearComando}{Prototipo crear comando.}{1.0}

\imagenAncho{crearPlantilla}{Prototipo crear plantilla.}{1.0}

\imagenAncho{asignarComandos}{Prototipo asignación de comandos en plantillas.}{1.0}

\imagenAncho{asignarPlantilla}{Prototipo asignación de plantillas.}{1.0}

El diseño final se ha realizado con la ayuda de Tailwind CSS y DaisyUI, ya que proporcionan una base sólida para el
diseño de la interfaz, permitiendo una personalización rápida y sencilla.
El resultado puede consultarse en el apartado de \textit{Manual del usuario}, donde se muestran ejemplos de las
pantallas del sistema.

Si desea obtener más información de cómo funciona Tailwind CSS, puede consultar su documentación oficial en
\url{https://tailwindcss.com/docs/installation} y del mismo modo, para DaisyUI en
\url{https://daisyui.com/docs/install/}.