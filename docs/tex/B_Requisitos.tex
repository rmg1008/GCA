\apendice{Especificación de Requisitos}

\section{Introducción}\label{sec:introduccion}

En esta sección se especifican los distintos requisitos del proyecto.
Este punto es clave porque define qué debe hacer el sistema incluso antes de comenzar con el desarrollo.

\section{Objetivos generales}\label{sec:objetivos-generales}

	\begin{itemize}
	\item Crear una plataforma para el control de acceso a los recursos de los dispositivos del centro.
	\item Contribuir a la automatización de procesos de la Universidad.
	\item Solucionar las necesidades del personal docente.
	\item Guardar las distintas configuraciones y permitir una gestión eficiente.
	\end{itemize}

\section{Catálogo de requisitos}\label{sec:catalogo-de-requisitos}

A continuación, se muestra el listado de los requisitos específicos del proyecto.

\subsection{Requisitos funcionales}\label{subsec:requisitos-funcionales}

\begin{itemize}
	\tightlist
	\item \textbf{RF-1 Gestión de la plataforma web de configuraciones:} La aplicación debe ser capaz de gestionar las configuraciones de los distintos equipos.
	\begin{itemize}
		\tightlist
		\item \textbf{RF-1.1 Registrar equipos:} El usuario debe poder dar de alta los ordenadores, mediante un identificador que permita validar la veracidad del equipo.
		\item \textbf{RF-1.2 Editar equipos:} La aplicación podrá permitir modificar datos de los equipos, así como sus configuraciones.
		\item \textbf{RF-1.3 Eliminar equipos:} El usuario debe poder dar de baja equipos que no se deseen seguir gestionando y eliminar la información asociada.
		\item \textbf{RF-1.4 Listar equipos:} La aplicación debe ser capaz de realizar listados de equipos y agruparlos atendiendo a diferentes características para después aplicar configuraciones pertinentes.
		\item \textbf{RF-1.5 Ver equipos:} La aplicación debe ser capaz de mostrar la configuración que aplica a un determinado equipo.
	\end{itemize}
	\item \textbf{RF-2 Gestión de plantillas de configuraciones:} La aplicación debe ser capaz de ofrecer una gestión eficiente de las plantillas de configuraciones que se aplicarán a los equipos.
	\begin{itemize}
		\tightlist
		\item \textbf{RF-2-1 Añadir plantillas:} El usuario debe poder añadir plantillas de configuraciones para los equipos.
		\item \textbf{RF-2.2 Modificar plantillas:} El usuario debe poder modificar plantillas existentes.
		\item \textbf{RF-2.3 Eliminar plantillas:} El usuario podrá eliminar plantillas que no vayan a usarse más.
		\item \textbf{RF-2.4 Listar plantillas:} El usuario podrá consultar las plantillas existentes.
		\item \textbf{RF-2.5 Ver plantillas:} El usuario podrá consultar toda la información perteneciente a una plantilla.
	\end{itemize}
	\item \textbf{RF-3 Gestión de usuarios:} La aplicación tiene que ser capaz de mantener una gestión de usuarios.
	\begin{itemize}
		\tightlist
		\item \textbf{RF-3.1 Añadir usuarios:} La aplicación permitirá dar de alta a nuevos usuarios que gestionen la aplicación.
		\item \textbf{RF-3.2 Modificar usuarios:} Se podrán modificar atributos de usuarios ya existentes.
		\item \textbf{RF-3.3 Eliminar usuarios:} La aplicación permitirá dar de baja a un usuario junto con sus datos existentes.
		\item \textbf{RF-3.4 Listar usuarios:} La aplicación podrá listar los usuarios que existen en el sistema.
		\item \textbf{RF-3.5 Ver usuarios:} La aplicación debe mostrar toda la información relativa al usuario.
	\end{itemize}
	\item \textbf{RF-4 Aplicativo de configuraciones:} La aplicación tiene que ser capaz de ejecutarse en el sistema operativo y aplicar en los equipos las configuraciones necesarias que correspondan a cada uno.
	\begin{itemize}
		\tightlist
		\item \textbf{RF-4.1 Registrar dispositivo:} La aplicación debe realizar el registro del dispositivo antes de poder aplicar las configuraciones.
		Se debe identificar de manera única a cada equipo.
		\item \textbf{RF-4.2 Aplicar configuración:} La aplicación tiene que ser capaz de aplicar la configuración registrada para cada dispositivo.
		\item \textbf{RF-4.3 Eliminar configuración:} La aplicación tiene que ser capaz de revertir los cambios realizados a un dispositivo.
	\end{itemize}
	\item 	\textbf{RF-5 Ayuda de la aplicación:} El usuario podrá obtener una ayuda acerca de cómo utilizar las funciones que se ofrecen en la aplicación.
\end{itemize}


\subsection{Requisitos no funcionales}\label{subsec:requisitos-no-funcionales}

Los requisitos no funcionales definen cómo debe comportarse un sistema.

\begin{itemize}
	\tightlist
	\item \textbf{RF-1 Rendimiento:} El sistema debe ofrecer unos tiempos de respuesta inferiores a 3 segundos bajo una carga media y ser capaz de escalar sin una degradación significativa del rendimiento.
	\item \textbf{RF-2 Seguridad:} La aplicación debe restringir el acceso a usuarios autorizados y cifrar datos sensibles, cumpliendo con buenas prácticas de seguridad.
	\item \textbf{RF-3 Usabilidad:} La aplicación debe ofrecer una interfaz intuitiva y accesible, garantizando una experiencia consistente y cumpliendo con unos principios básicos de accesibilidad.
	\item \textbf{RF-4 Disponibilidad:} La aplicación debe estar operativa de forma continua, a excepción de situaciones extraordinarias o mantenimientos programados.
	\item \textbf{RF-5 Mantenibilidad:} La aplicación debe seguir unos estándares y buenas prácticas, facilitando su mantenimiento, evolución y comprensión por parte de otros desarrolladores.
	\item \textbf{RF-6 Portabilidad:} La aplicación debe ser portable entre distintas plataformas, utilizando tecnologías que faciliten su migración, como pueden ser contenedores.
	\item \textbf{RF-7 Legalidad:} El sistema debe hacer un uso adecuado de herramientas de terceros y cumplir con las normas del territorio en el que se aloja.
\end{itemize}

\section{Especificación de requisitos}\label{sec:especificacion-de-requisitos}

\subsection{Casos de uso}\label{subsec:casos-de-uso}

% Caso de Uso 1.2 -> Registrar equipos.
\begin{table}[p]
	\centering
	\begin{tabularx}{\linewidth}{ p{0.21\columnwidth} p{0.71\columnwidth} }
		\toprule
		\textbf{CU-1}    & \textbf{ Registrar equipos}\\
		\toprule
		\textbf{Versión}              & 1.0    \\
		\textbf{Autor}                & \@author{} \\
		\textbf{Requisitos asociados} & RF-1.1\\
		\textbf{Descripción}          & Permite al usuario dar de alta un nuevo equipo. \\
		\textbf{Precondición}         &
		\begin{itemize}
			\tightlist
			\item La aplicación se encuentra operativa.
			\item La base de datos es correcta.
			\item El usuario está autentificado en el sistema.
			\item El usuario tiene suficientes permisos para realizar la operación.
		\end{itemize}\\
		\textbf{Acciones}             &
		\begin{enumerate}
			\tightlist
			\item Acceder al área de gestión de equipos y presionar en añadir.
			\item Rellenar el formulario (huella digital, nombre del equipo\ldots).
			\item Pulsar en guardar información.
		\end{enumerate}\\
		\textbf{Postcondición}        & El equipo se encuentra registrado en la BD.\\
		\textbf{Excepciones}          &
		\begin{itemize}
			\tightlist
			\item La huella digital o el nombre ya existe.
			\item No se ha introducido algún campo obligatorio (se muestra mensaje).
		\end{itemize}\\
		\textbf{Importancia}          & Alta\\
		\bottomrule
	\end{tabularx}
	\caption{CU-1 Registrar equipos.}\label{tab:table-1}
\end{table}

\begin{table}[p]
	\centering
	\begin{tabularx}{\linewidth}{ p{0.21\columnwidth} p{0.71\columnwidth} }
		\toprule
		\textbf{CU-2}    & \textbf{ Editar equipos}\\
		\toprule
		\textbf{Versión}              & 1.0    \\
		\textbf{Autor}                & \@author{} \\
		\textbf{Requisitos asociados} & RF-1.2\\
		\textbf{Descripción}          & Permite al usuario modificar datos de un equipo. \\
		\textbf{Precondición}         &
		\begin{itemize}
			\tightlist
			\item La aplicación se encuentra operativa.
			\item La base de datos es correcta.
			\item El usuario está autentificado en el sistema y el equipo debe existir.
			\item El usuario tiene suficientes permisos para realizar la operación.
		\end{itemize}\\
		\textbf{Acciones}             &
		\begin{enumerate}
			\tightlist
			\item Escoger el equipo que se quiere modificar.
			\item Rellenar los campos necesarios (huella digital, nombre del equipo\ldots).
			\item Pulsar en modificar información.
		\end{enumerate}\\
		\textbf{Postcondición}        & El equipo se encuentra actualizado en la BD.\\
		\textbf{Excepciones}          &
		\begin{itemize}
			\tightlist
			\item La huella digital o el nombre ya existe.
			\item No se ha introducido algún campo obligatorio (se muestra mensaje).
			\item El equipo a editar no existe.
		\end{itemize}\\
		\textbf{Importancia}          & Alta\\
		\bottomrule
	\end{tabularx}
	\caption{CU-2 Editar equipos.}\label{tab:table-2}
\end{table}

\begin{table}[p]
	\centering
	\begin{tabularx}{\linewidth}{ p{0.21\columnwidth} p{0.71\columnwidth} }
		\toprule
		\textbf{CU-3}    & \textbf{ Eliminar equipos}\\
		\toprule
		\textbf{Versión}              & 1.0    \\
		\textbf{Autor}                & \@author{} \\
		\textbf{Requisitos asociados} & RF-1.3\\
		\textbf{Descripción}          & Permite al usuario dar de baja equipo. \\
		\textbf{Precondición}         &
		\begin{itemize}
			\tightlist
			\item La aplicación se encuentra operativa.
			\item La base de datos es correcta.
			\item El usuario está autentificado en el sistema y el equipo debe existir.
			\item El usuario tiene suficientes permisos para realizar la operación.
		\end{itemize}\\
		\textbf{Acciones}             &
		\begin{enumerate}
			\tightlist
			\item Escoger el equipo que se quiere eliminar.
			\item Pulsar en eliminar equipo.
			\item Confirmar eliminación.
		\end{enumerate}\\
		\textbf{Postcondición}        & El equipo se encuentra eliminado de la BD.\\
		\textbf{Excepciones}          &
		\begin{itemize}
			\tightlist
			\item No se ha podido borrar el registro (se muestra mensaje).
		\end{itemize}\\
		\textbf{Importancia}          & Alta\\
		\bottomrule
	\end{tabularx}
	\caption{CU-3 Eliminar equipos.}\label{tab:table-3}
\end{table}

\begin{table}[p]
	\centering
	\begin{tabularx}{\linewidth}{ p{0.21\columnwidth} p{0.71\columnwidth} }
		\toprule
		\textbf{CU-4}    & \textbf{ Ver y listar equipos}\\
		\toprule
		\textbf{Versión}              & 1.0    \\
		\textbf{Autor}                & \@author{} \\
		\textbf{Requisitos asociados} & RF-1.4, RF-1.5\\
		\textbf{Descripción}          & Permite al usuario obtener listados de equipos con determinados filtros. \\
		\textbf{Precondición}         &
		\begin{itemize}
			\tightlist
			\item La aplicación se encuentra operativa.
			\item La base de datos es correcta.
			\item El usuario está autentificado en el sistema.
			\item El usuario tiene suficientes permisos para realizar la operación.
		\end{itemize}\\
		\textbf{Acciones}             &
		\begin{enumerate}
			\tightlist
			\item Acceder al área de gestión de equipos.
			\item Escoger los distintos filtros para realizar la búsqueda.
			\item Presionar en buscar.
			\item Seleccionar un equipo entre los encontrados para ver sus detalles.
		\end{enumerate}\\
		\textbf{Postcondición}        & Se muestran correctamente los equipos, si se cumplen los criterios de búsqueda.\\
		\textbf{Excepciones}          &
		\begin{itemize}
			\tightlist
			\item Se muestra un texto si no se han encontrado equipos para esos criterios.
		\end{itemize}\\
		\textbf{Importancia}          & Alta\\
		\bottomrule
	\end{tabularx}
	\caption{CU-4 Ver y listar equipos.}\label{tab:table-4}
\end{table}

\begin{table}[p]
	\centering
	\begin{tabularx}{\linewidth}{ p{0.21\columnwidth} p{0.71\columnwidth} }
		\toprule
		\textbf{CU-5}    & \textbf{ Añadir plantillas}\\
		\toprule
		\textbf{Versión}              & 1.0    \\
		\textbf{Autor}                & \@author{} \\
		\textbf{Requisitos asociados} & RF-2.1\\
		\textbf{Descripción}          & Permite al usuario crear plantillas de configuración que se aplicarán a los equipos.\\
		\textbf{Precondición}         &
		\begin{itemize}
			\tightlist
			\item La aplicación se encuentra operativa.
			\item La base de datos es correcta.
			\item El usuario está autentificado en el sistema.
			\item El usuario tiene suficientes permisos para realizar la operación.
		\end{itemize}\\
		\textbf{Acciones}             &
		\begin{enumerate}
			\tightlist
			\item Acceder al área de gestión de plantillas y presionar en añadir.
			\item Introducir los datos de la plantilla (nombre, descripción\dots) y los comandos a aplicar sobre ella.
			\item Presionar en guardar.
		\end{enumerate}\\
		\textbf{Postcondición}        & La plantilla se almacena correctamente en base de datos.\\
		\textbf{Excepciones}          &
		\begin{itemize}
			\tightlist
			\item Se muestra un mensaje si no se tienen permisos para almacenar la plantilla.
		\end{itemize}\\
		\textbf{Importancia}          & Alta\\
		\bottomrule
	\end{tabularx}
	\caption{CU-5 Añadir plantillas.}\label{tab:table-5}
\end{table}

\begin{table}[p]
	\centering
	\begin{tabularx}{\linewidth}{ p{0.21\columnwidth} p{0.71\columnwidth} }
		\toprule
		\textbf{CU-6}    & \textbf{ Modificar plantillas}\\
		\toprule
		\textbf{Versión}              & 1.0    \\
		\textbf{Autor}                & \@author{} \\
		\textbf{Requisitos asociados} & RF-2.2\\
		\textbf{Descripción}          & Permite al usuario crear plantillas de configuración que se aplicarán a los equipos.\\
		\textbf{Precondición}         &
		\begin{itemize}
			\tightlist
			\item La aplicación se encuentra operativa.
			\item La base de datos es correcta.
			\item El usuario está autentificado en el sistema.
			\item El usuario tiene suficientes permisos para realizar la operación.
			\item La plantilla existe.
		\end{itemize}\\
		\textbf{Acciones}             &
		\begin{enumerate}
			\tightlist
			\item Acceder al área de gestión de plantillas y seleccionar aquella que se quiere modificar.
			\item Introducir los datos modificados de la plantilla (nombre, descripción\dots) y los comandos sobre ella.
			\item Presionar en modificar.
		\end{enumerate}\\
		\textbf{Postcondición}        & La plantilla se modifica correctamente en base de datos.\\
		\textbf{Excepciones}          &
		\begin{itemize}
			\tightlist
			\item Se muestra un mensaje si no se tienen permisos para modificar la plantilla.
		\end{itemize}\\
		\textbf{Importancia}          & Media\\
		\bottomrule
	\end{tabularx}
	\caption{CU-6 Modificar plantillas.}\label{tab:table-6}
\end{table}

\begin{table}[p]
	\centering
	\begin{tabularx}{\linewidth}{ p{0.21\columnwidth} p{0.71\columnwidth} }
		\toprule
		\textbf{CU-7}    & \textbf{ Eliminar plantillas}\\
		\toprule
		\textbf{Versión}              & 1.0    \\
		\textbf{Autor}                & \@author{} \\
		\textbf{Requisitos asociados} & RF-2.3\\
		\textbf{Descripción}          & Permite al usuario dar de baja plantillas.\\
		\textbf{Precondición}         &
		\begin{itemize}
			\tightlist
			\item La aplicación se encuentra operativa.
			\item La base de datos es correcta.
			\item El usuario está autentificado en el sistema.
			\item El usuario tiene suficientes permisos para realizar la operación.
			\item La plantilla existe.
		\end{itemize}\\
		\textbf{Acciones}             &
		\begin{enumerate}
			\tightlist
			\item Acceder al área de gestión de plantillas y seleccionar aquella que se quiere eliminar.
			\item Presionar en eliminar.
			\item Confirmar eliminación.
		\end{enumerate}\\
		\textbf{Postcondición} &
		\begin{itemize}
			\tightlist
			\item La plantilla se elimina del sistema.
			\item No se vuelve a aplicar la plantilla eliminada sobre los equipos asociados.
		\end{itemize}\\
		\textbf{Excepciones}          &
		\begin{itemize}
			\tightlist
			\item Se muestra un mensaje si no se tienen permisos para eliminar la plantilla.
		\end{itemize}\\
		\textbf{Importancia}          & Media\\
		\bottomrule
	\end{tabularx}
	\caption{CU-7 Eliminar plantillas.}\label{tab:table-7}
\end{table}

\begin{table}[p]
	\centering
	\begin{tabularx}{\linewidth}{ p{0.21\columnwidth} p{0.71\columnwidth} }
		\toprule
		\textbf{CU-8}    & \textbf{ Ver y listar plantillas}\\
		\toprule
		\textbf{Versión}              & 1.0    \\
		\textbf{Autor}                & \@author{} \\
		\textbf{Requisitos asociados} & RF-2.4, RF-2.5\\
		\textbf{Descripción}          & Permite al usuario obtener listados de plantillas con determinados filtros. \\
		\textbf{Precondición}         &
		\begin{itemize}
			\tightlist
			\item La aplicación se encuentra operativa.
			\item La base de datos es correcta.
			\item El usuario está autentificado en el sistema.
			\item El usuario tiene suficientes permisos para realizar la operación.
		\end{itemize}\\
		\textbf{Acciones}             &
		\begin{enumerate}
			\tightlist
			\item Acceder al área de gestión de plantillas.
			\item Escoger los distintos filtros para realizar la búsqueda.
			\item Presionar en buscar.
			\item Seleccionar una plantilla entre las encontradas para ver sus detalles.
		\end{enumerate}\\
		\textbf{Postcondición}        & Se muestran correctamente las plantillas, si se cumplen los criterios de búsqueda.\\
		\textbf{Excepciones}          &
		\begin{itemize}
			\tightlist
			\item Se muestra un texto si no se han encontrado plantillas para esos criterios.
		\end{itemize}\\
		\textbf{Importancia}          & Alta\\
		\bottomrule
	\end{tabularx}
	\caption{CU-8 Ver y listar equipos.}\label{tab:table-8}
\end{table}

\begin{table}[p]
	\centering
	\begin{tabularx}{\linewidth}{ p{0.21\columnwidth} p{0.71\columnwidth} }
		\toprule
		\textbf{CU-9}    & \textbf{ Añadir usuarios}\\
		\toprule
		\textbf{Versión}              & 1.0    \\
		\textbf{Autor}                & \@author{} \\
		\textbf{Requisitos asociados} & RF-3-1\\
		\textbf{Descripción}          & Permite dar de alta usuarios en el sistema con distintos roles. \\
		\textbf{Precondición}         &
		\begin{itemize}
			\tightlist
			\item La aplicación se encuentra operativa.
			\item La base de datos es correcta.
		\end{itemize}\\
		\textbf{Acciones}             &
		\begin{enumerate}
			\tightlist
			\item Acceder al área de gestión de usuarios.
			\item Seleccionar dar de alta un usuario.
			\item Rellenar la información del usuario.
			\item Pulsar en aceptar.
			\item Si no hay ningún error, se envía un mail de confirmación para verificar la cuenta.
		\end{enumerate}\\
		\textbf{Postcondición}        & El usuario ha validado su cuenta y se encuentra registrado en la BD.\\
		\textbf{Excepciones}          &
		\begin{itemize}
			\tightlist
			\item Si se ha introducido algún dato único ya almacenado (por ejemplo, email) se muestra un mensaje.
		\end{itemize}\\
		\textbf{Importancia}          & Alta\\
		\bottomrule
	\end{tabularx}
	\caption{CU-9 Añadir usuarios.}\label{tab:table-9}
\end{table}

\begin{table}[p]
	\centering
	\begin{tabularx}{\linewidth}{ p{0.21\columnwidth} p{0.71\columnwidth} }
		\toprule
		\textbf{CU-10}    & \textbf{ Modificar usuarios}\\
		\toprule
		\textbf{Versión}              & 1.0    \\
		\textbf{Autor}                & \@author{} \\
		\textbf{Requisitos asociados} & RF-3-2\\
		\textbf{Descripción}          & Permite modificar datos relativos al usuario. \\
		\textbf{Precondición}         &
		\begin{itemize}
			\tightlist
			\item La aplicación se encuentra operativa.
			\item La base de datos es correcta.
			\item El usuario se encuentra previamente creado.
			\item El usuario tiene suficientes permisos para realizar la operación.
		\end{itemize}\\
		\textbf{Acciones}             &
		\begin{enumerate}
			\tightlist
			\item Acceder al área de gestión de usuarios.
			\item Seleccionar modificar usuario.
			\item Rellenar la información que se quiere modificar y puede actualizarse (por ejemplo, el correo electrónico no).
			\item Pulsar en guardar.
			\item Si no hay ningún error, se actualiza la cuenta.
		\end{enumerate}\\
		\textbf{Postcondición}        & El usuario ha sido actualizado en la BD.\\
		\textbf{Excepciones}          &
		\begin{itemize}
			\tightlist
			\item Si se ha introducido algún dato único ya almacenado (por ejemplo, email) se muestra un mensaje.
		\end{itemize}\\
		\textbf{Importancia}          & Media\\
		\bottomrule
	\end{tabularx}
	\caption{CU-10 Modificar usuarios.}\label{tab:table-10}
\end{table}

\begin{table}[p]
	\centering
	\begin{tabularx}{\linewidth}{ p{0.21\columnwidth} p{0.71\columnwidth} }
		\toprule
		\textbf{CU-11}    & \textbf{ Eliminar usuarios}\\
		\toprule
		\textbf{Versión}              & 1.0    \\
		\textbf{Autor}                & \@author{} \\
		\textbf{Requisitos asociados} & RF-3-3\\
		\textbf{Descripción}          & Permite dar de baja un usuario. \\
		\textbf{Precondición}         &
		\begin{itemize}
			\tightlist
			\item La aplicación se encuentra operativa.
			\item La base de datos es correcta.
			\item El usuario tiene suficientes permisos para realizar la operación.
		\end{itemize}\\
		\textbf{Acciones}             &
		\begin{enumerate}
			\tightlist
			\item Acceder al área de gestión de usuarios.
			\item Seleccionar el usuario que se desea eliminar.
			\item Pulsar en eliminar cuenta.
			\item Confirmar la eliminación.
			\item Si no hay ningún error, se elimina la cuenta.
		\end{enumerate}\\
		\textbf{Postcondición}        & El usuario ha sido eliminado de la BD.\\
		\textbf{Excepciones}          &
		\begin{itemize}
			\tightlist
			\item Si no se ha podido eliminar se muestra un mensaje.
		\end{itemize}\\
		\textbf{Importancia}          & Media\\
		\bottomrule
	\end{tabularx}
	\caption{CU-11 Eliminar usuarios.}\label{tab:table-11}
\end{table}

\begin{table}[p]
	\centering
	\begin{tabularx}{\linewidth}{ p{0.21\columnwidth} p{0.71\columnwidth} }
		\toprule
		\textbf{CU-12}    & \textbf{ Ver y listar usuarios}\\
		\toprule
		\textbf{Versión}              & 1.0    \\
		\textbf{Autor}                & \@author{} \\
		\textbf{Requisitos asociados} & RF-3.4, RF-3.5\\
		\textbf{Descripción}          & Permite obtener listados de usuarios con determinados filtros. \\
		\textbf{Precondición}         &
		\begin{itemize}
			\tightlist
			\item La aplicación se encuentra operativa.
			\item La base de datos es correcta.
			\item El usuario tiene suficientes permisos para realizar la operación.
		\end{itemize}\\
		\textbf{Acciones}             &
		\begin{enumerate}
			\tightlist
			\item Acceder al área de gestión de usuarios.
			\item Escoger los distintos filtros para realizar la búsqueda.
			\item Presionar en buscar.
			\item Seleccionar aquel usuario entre los encontrados para ver sus detalles.
		\end{enumerate}\\
		\textbf{Postcondición}        & Se muestran correctamente las usuarios, si se cumplen los criterios de búsqueda.\\
		\textbf{Excepciones}          &
		\begin{itemize}
			\tightlist
			\item Se muestra un texto si no se han encontrado usuarios para esos criterios.
		\end{itemize}\\
		\textbf{Importancia}          & Alta\\
		\bottomrule
	\end{tabularx}
	\caption{CU-12 Ver y listar usuarios.}\label{tab:table-12}
\end{table}

\begin{table}[p]
	\centering
	\begin{tabularx}{\linewidth}{ p{0.21\columnwidth} p{0.71\columnwidth} }
		\toprule
		\textbf{CU-13}    & \textbf{ Registrar dispositivo}\\
		\toprule
		\textbf{Versión}              & 1.0    \\
		\textbf{Autor}                & \@author{} \\
		\textbf{Requisitos asociados} & RF-4.1\\
		\textbf{Descripción}          & Permite realizar un registro del dispositivo para aplicar futuras configuraciones. \\
		\textbf{Precondición}         &
		\begin{itemize}
			\tightlist
			\item La aplicación se encuentra operativa.
			\item La base de datos es correcta.
			\item El usuario posee el ejecutable que permite iniciar el registro.
			\item El usuario tiene suficientes permisos para realizar la operación.
		\end{itemize}\\
		\textbf{Acciones}             &
		\begin{enumerate}
			\tightlist
			\item El usuario ejecuta el programa.
			\item Se solicitan credenciales para comprobar si se tienen permisos de registro.
			\item Una vez introducidos y validados, el proceso continúa.
			\item Se obtiene la huella digital para ese dispositivo, se almacena en el SO y se envía a la aplicación para su registro.
			\item Se programa una tarea en la que la herramienta se ejecuta periódicamente.
		\end{enumerate}\\
		\textbf{Postcondición}        & Se registra correctamente el dispositivo y se establece una tarea periódica en el sistema.\\
		\textbf{Excepciones}          &
		\begin{itemize}
			\tightlist
			\item Error de usuario (no tiene suficientes permisos).
			\item Error de registro (se muestra un mensaje).
		\end{itemize}\\
		\textbf{Importancia}          & Alta\\
		\bottomrule
	\end{tabularx}
	\caption{CU-13 Registrar dispositivo.}\label{tab:table-13}
\end{table}

\begin{table}[p]
	\centering
	\begin{tabularx}{\linewidth}{ p{0.21\columnwidth} p{0.71\columnwidth} }
		\toprule
		\textbf{CU-14}    & \textbf{ Aplicar configuración}\\
		\toprule
		\textbf{Versión}              & 1.0    \\
		\textbf{Autor}                & \@author{} \\
		\textbf{Requisitos asociados} & RF-4.2\\
		\textbf{Descripción}          & Permite aplicar configuraciones a determinados dispositivos. \\
		\textbf{Precondición}         &
		\begin{itemize}
			\tightlist
			\item La aplicación se encuentra operativa.
			\item La base de datos es correcta.
			\item El dispositivo se encuentra registrado en el sistema y se le ha asignado una configuración.
			\item Existe una tarea programada en el dispositivo.
		\end{itemize}\\
		\textbf{Acciones}             &
		\begin{enumerate}
			\tightlist
			\item El dispositivo ejecuta la tarea programada.
			\item Se comprueba que la huella digital esté registrada en el sistema.
			\item Si no se encuentra registrada, se realiza el registro, en caso contrario, continúa el proceso.
			\item Se obtiene la configuración a aplicar para el dispositivo.
			\item Se aplica la configuración para el dispositivo.
		\end{enumerate}\\
		\textbf{Postcondición}        & Se aplica correctamente la configuración almacenada para el dispositivo.\\
		\textbf{Excepciones}          &
		\begin{itemize}
			\tightlist
			\item Error al aplicar la configuración (se muestra mensaje).
		\end{itemize}\\
		\textbf{Importancia}          & Alta\\
		\bottomrule
	\end{tabularx}
	\caption{CU-14 Aplicar configuración.}\label{tab:table-14}
\end{table}

\begin{table}[p]
	\centering
	\begin{tabularx}{\linewidth}{ p{0.21\columnwidth} p{0.71\columnwidth} }
		\toprule
		\textbf{CU-15}    & \textbf{ Eliminar configuración}\\
		\toprule
		\textbf{Versión}              & 1.0    \\
		\textbf{Autor}                & \@author{} \\
		\textbf{Requisitos asociados} & RF-4.3\\
		\textbf{Descripción}          & Permite eliminar configuraciones de los dispositivos. \\
		\textbf{Precondición}         &
		\begin{itemize}
			\tightlist
			\item La aplicación se encuentra operativa.
			\item La base de datos es correcta.
			\item El dispositivo se encuentra registrado en el sistema y se ha dado de baja una configuración.
			\item Existe una tarea programada en el dispositivo.
		\end{itemize}\\
		\textbf{Acciones}             &
		\begin{enumerate}
			\tightlist
			\item El dispositivo ejecuta la tarea programada.
			\item Se comprueba que la huella digital esté registrada en el sistema.
			\item Si no se encuentra registrada, se realiza el registro, en caso contrario, continúa el proceso.
			\item Se obtiene que hay que eliminar la configuración.
			\item Se resetea la configuración para el dispositivo.
		\end{enumerate}\\
		\textbf{Postcondición}        & Se resetea correctamente la configuración del dispositivo.\\
		\textbf{Excepciones}          &
		\begin{itemize}
			\tightlist
			\item Error al aplicar la configuración (se muestra mensaje).
		\end{itemize}\\
		\textbf{Importancia}          & Alta\\
		\bottomrule
	\end{tabularx}
	\caption{CU-15 Eliminar configuración.}\label{tab:table-15}
\end{table}

\begin{table}[p]
	\centering
	\begin{tabularx}{\linewidth}{ p{0.21\columnwidth} p{0.71\columnwidth} }
		\toprule
		\textbf{CU-16}    & \textbf{ Ayuda de la aplicación}\\
		\toprule
		\textbf{Versión}              & 1.0    \\
		\textbf{Autor}                & \@author{} \\
		\textbf{Requisitos asociados} & RF-5\\
		\textbf{Descripción}          & Permite ofrecer ayuda a los usuarios de la aplicación. \\
		\textbf{Precondición}         &
		\begin{itemize}
			\tightlist
			\item La aplicación se encuentra operativa.
		\end{itemize}\\
		\textbf{Acciones}             &
		\begin{enumerate}
			\tightlist
			\item El usuario pulsa en ayuda de la aplicación.
			\item Se abre un manual de ayuda para el usuario.
		\end{enumerate}\\
		\textbf{Postcondición}        & El usuario obtiene ayuda acerca de la aplicación.\\
		\textbf{Excepciones}          &
		\begin{itemize}
			\tightlist
			\item Error al generar la ayuda (se muestra mensaje).
		\end{itemize}\\
		\textbf{Importancia}          & Baja\\
		\bottomrule
	\end{tabularx}
	\caption{CU-16 Ayuda de la aplicación.}\label{tab:table-16}
\end{table}
