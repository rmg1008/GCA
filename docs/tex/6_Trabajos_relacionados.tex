\capitulo{6}{Trabajos relacionados}

La gestión centralizada de dispositivos es una práctica habitual en la administración de sistemas
y, casi desde sus inicios, se han desarrollado herramientas para la gestión de dispositivos de red.
Esta gestión ha evolucionado (y sigue evolucionando) hacia la automatización de tareas, la orquestación de servicios y
la gestión de configuraciones.

En este apartado se exponen aquellos artículos científicos y herramientas existentes que guardan similitud con
el presente trabajo.

\section{Articulos cientificos}\label{sec:articulos-cientificos}

\textbf{An Application to Automate the Configuration Management and Assessment of Practically Based Learning Outcomes in Computer Networking }\cite{malliarakis2019application}

Este estudio describe una aplicación desarrollada para automatizar la configuración y evaluación de escenarios prácticos
en laboratorios de redes.
La herramienta permite aplicar configuraciones a dispositivos reales y simulados, facilitando  la evaluación de
resultados de aprendizaje prácticos.
Aunque se centra en el ámbito educativo, su enfoque en la automatización y gestión de configuraciones es aplicable a
entornos universitarios más amplios.

\textbf{Configuration Management and Security }\cite{bellovin2001configuration}

Este artículo analiza los desafíos de la gestión de configuración en sistemas de gran
escala, destacando la importancia de políticas centralizadas para garantizar la seguridad y el cumplimiento normativo.
Aunque no se enfoca específicamente en la educación, se describe la necesidad de herramientas que permitan la gestión
de configuraciones de manera eficiente y segura.

\section{Herramientas}\label{sec:herramientas}

\textbf{System Center Configuration Manager (SCCM)}

SCCM\cite{microsoftConfigMgr} es una solución de Microsoft para la gestión centralizada de configuraciones y
actualizaciones.
Permite aplicar configuraciones a grupos de dispositivos, gestionar actualizaciones de software \ldots
Aunque es una herramienta comercial, su enfoque en la gestión de configuraciones es similar al
propuesto en este trabajo.
Para su uso es necesario tener instalado Microsoft AD para permitir gestionar dispositivos y usuarios dentro
de la red.

\textbf{Active Directory (AD) o Entra ID}

AD\cite{microsoftADDS} y Entra ID\cite{microsoftEntra} (anteriormente conocido como Azure AD) son soluciones de Microsoft
para la gestión centralizada mediante servicios que controlan el acceso a recursos y la configuración de dispositivos
en entornos empresariales.
La diferencia principal entre ambos es que Entra ID está diseñado para entornos en la nube, mientras que AD se utiliza
principalmente en entornos locales.

\textbf{Netsh}

Netsh\cite{microsoftNetsh} es una herramienta de comandos de Microsoft que permite la configuración y gestión
de dispositivos de red en sistemas operativos Windows, ya sea de forma local o remota.

 \begin{table}[H]
        \centering
        \resizebox{\textwidth}{!}{%
        \begin{tabular}{lcccc}
            \toprule
            \textbf{Características} & \textbf{GCA} & \textbf{SCCM} & \textbf{AD} & \textbf{Netsh} \\
            \midrule
            Automatización de configuraciones & \cellcolor{verde}Sí & \cellcolor{verde}Sí & \cellcolor{verde}Sí & \cellcolor{rojo}No \\
            Gestión centralizada              & \cellcolor{verde}Sí & \cellcolor{verde}Sí & \cellcolor{verde}Sí & \cellcolor{rojo}No \\
            Priorización por dispositivos     & \cellcolor{verde}Si & \cellcolor{amarillo}Limitada & \cellcolor{verde}Sí & \cellcolor{rojo}No \\
            Interfaz de usuario               & \cellcolor{verde}Sí & \cellcolor{verde}Sí & \cellcolor{verde}Sí & \cellcolor{rojo}No (terminal) \\
            Curva de aprendizaje              & \cellcolor{verde}Baja & \cellcolor{rojo}Alta & \cellcolor{rojo}Alta & \cellcolor{amarillo}Media \\
            Dependencia con otras herramientas & \cellcolor{verde}No & \cellcolor{rojo}Sí & \cellcolor{rojo}Sí & \cellcolor{verde}No \\
            Multiplataforma                   & \cellcolor{verde}Sí* & \cellcolor{amarillo}Limitada & \cellcolor{amarillo}Limitada & \cellcolor{rojo}No \\
            \bottomrule
        \end{tabular}
        }
        \caption{Comparativa de las herramientas de los proyectos con codificación de colores.}
        \label{tab:comparativa-herramientas-proyectos}
 \end{table}

Esta comparativa revela que, aunque soluciones como SCCM y Active Directory ofrecen capacidades avanzadas de gestión de
configuraciones y escalabilidad, no resultan especialmente rentables para el fin que pretendemos porque requieren de
hardware específico, son económicamente costosas y están diseñadas para entornos empresariales.
Por el contrario, la propuesta de GCA se centra en la simplicidad y accesibilidad, permitiendo su uso en entornos educativos
y de investigación sin la necesidad de infraestructura costosa y compleja, pudiendo ser implementada en cualquier
tipo entorno, si así se desea.

No obstante, la elección de la herramienta adecuada dependerá de las necesidades específicas de cada entorno y de los
recursos disponibles.

\section{Fortalezas y debilidades del proyecto}\label{sec:fortalezas-y-debilidades-del-proyecto}

\begin{table}[ht]
    \centering
    \begin{tabular}{@{}p{6cm} p{6cm}@{}}
        \toprule
        \textbf{Fortalezas} & \textbf{Debilidades} \\
        \midrule
        Automatización de tareas repetitivas & Dependencia de la conectividad \\
        Gestión centralizada y escalable & Complejidad en la implementación \\
        Priorización de configuraciones & Riesgo de obsolescencia tecnológica \\
        Integración con herramientas existentes & Requiere primera instalación manual \\
        Mejora en la seguridad & Posibles problemas de compatibilidad \\
        Compatible con cualquier sistema operativo & Necesidad de mantenimiento continuo \\
        \bottomrule
    \end{tabular}
    \caption{Fortalezas y Debilidades del Proyecto}
    \label{tab:fortalezas-debilidades}
\end{table}