\apendice{Documentación de usuario}

\section{Introducción}

El público al que va dirigido este manual son los usuarios de la aplicación GCA: Administradores y Gestores.
Por lo que para entender su contenido, se asume que el lector tiene conocimientos en la administración de sistemas y
redes.
Este manual tiene como objetivo detallar todos aquellos aspectos que se consideran necesarios para los usuarios,
desde la instalación y sus requisitos hasta el uso de la aplicación.

\section{Requisitos de usuarios}

Es importante mencionar que GCA utiliza dos herramientas:
\begin{itemize}
    \item \textbf{Plataforma web}
    \item \textbf{Aplicación cliente}
\end{itemize}

\subsection{Plataforma web}

Para la utilización de la web, se requiere de:
\begin{itemize}
    \item Navegador web moderno (Google Chrome, Microsoft Edge o Firefox actualizados a su última versión).
    \item Conexión a internet estable.
    \item Dispositivo con una resolución mínima de 1280x720 píxeles.
    \item Acceso a la URL de la plataforma y credenciales proporcionadas por el administrador del sistema.
\end{itemize}

\subsection{Aplicación cliente}

Para la aplicación de escritorio, es necesario instalar el agente en un dispositivo con sistema operativo
Windows.
Los requisitos son:
\begin{itemize}
    \item Sistema operativo Windows (64 bits) o superior.
    \item Procesador con, al menos, dos núcleos para asegurar una ejecución fluida de los procesos en paralelo.
    \item Al menos 100 MB de espacio libre en el disco.
    \item Permisos de administrador.
    \item Conexión a internet para comunicarse con el servidor.
    \item Credenciales de acceso al sistema, las mismas que se utilizan en la web.
\end{itemize}

\section{Instalación}

\subsection{Plataforma web}

La plataforma web no requiere ninguna instalación, simplemente acceder a la URL proporcionada por el administrador
del sistema.

\subsection{Aplicación cliente}

Para ello, es necesario descargar el instalador.
Se puede acceder a él a través del siguiente
\href{https://github.com/rmg1008/GCA/blob/main/client/output/GCA_Installer.exe}{enlace} o utilizar el proporcionado
por el administrador del sistema.

\imagen{gca_installer.png}{Instalador aplicación cliente}{0.5}

Una vez descargado:
\begin{itemize}
    \item Clic derecho sobre el instalador y pulsamos en ``Ejecutar como administrador``: aparecerá un recuadro
    de confirmación para permitir su ejecución.
    \item Se abrirá un asistente de instalación.
    Escogemos la ruta para instalarlo: podemos utilizar la que viene por defecto y pulsamos en ``Siguiente``.
    \item Elegir dónde colocar el acceso directo y pulsar ``Siguiente``.
    \item Marcar la casilla si se quiere crear un acceso directo en el escritorio y pulsar ``Siguiente``.
    \item Por último, cuando se muestra el resumen de la configuración selecciona, pulsar ``Instalar``.
    \item Una vez instalado, se muestra una última ventana que nos permite lanzar directamente la aplicación.
\end{itemize}

\imagen{gca_installer_1.png}{Ventana para elegir la ruta de instalación}{0.7}

\imagen{gca_installer_2.png}{Ventana resumen de la configuración seleccionada}{0.7}

\imagen{gca_installer_3.png}{Finalización de la instalación}{0.7}

Siguiendo estos pasos, la aplicación se instala correctamente.
Sin embargo, es necesario realizar una última configuración para permitir su ejecución sin ser eliminada por Windows
Defender.

\begin{itemize}
    \item Acude a ``Seguridad de Windows`` - ``Protección contra virus y amenazas``
    \item Haz clic en ``Administrar configuración`` y busca ``Añadir o quitar exclusiones``.
    \item Añade la carpeta dónde se ha instalado la aplicación.
    \item Puedes obtener más información en el siguiente
    \href{https://consumer.huawei.com/es/support/content/es-es00694627/#:~:text=Acceda%20a%20Seguridad%20de%20Windows,o%20quitar%20exclusiones%20en%20Exclusiones.}
    {tutorial}.
\end{itemize}

Tras realizar la configuración, la aplicación se encuentra totalmente lista para ser ejecutada.

\imagen{gca_installer_4.png}{Exclusión de la aplicación GCA}{0.5}

\section{Manual del usuario}

Se detallan los procesos necesarios para la utilización de ambas herramientas.

\subsection{Plataforma web}

Toda la plataforma web se divide en diferentes secciones, cada una de ellas con un propósito específico.
\begin{enumerate}
    \item \textbf{Cabecera}.
    En la parte superior de la pantalla, se muestra el icono de la aplicación a la izquierda, y los
    enlaces a las diferentes secciones de la plataforma a la derecha.
    \item \textbf{Contenido principal}.
    En el centro de la pantalla, se muestra el contenido principal de la página
    seleccionada.
    Dependiendo de la página, se mostrarán diferentes elementos, como tablas, formularios, buscadores, etc.
    \item \textbf{Pie de página}.
    En la parte inferior de la pantalla, se muestra el nombre de la aplicación y la licencia
    bajo la que se distribuye.
\end{enumerate}

El acceso a la plataforma web se realiza mediante el inicio de sesión, donde se debe introducir el correo del usuario
y la contraseña.
\imagen{gca_web.png}{Página de Inicio de Sesión de la plataforma web}{1}

Tras introducir las credenciales correctamente, se muestra la página principal de la plataforma:

\imagen{dashboard_web.png}{Página principal de la plataforma web}{1}

Dentro del contenido principal, se observan diferentes secciones:
\begin{itemize}
    \item \textbf{Menú lateral izquierdo}.
    A la izquierda de la pantalla, se muestra un menú con los diferentes grupos que conforman la plataforma.
    El grupo raíz ``UBU`` contiene a todos los demás grupos, y siempre se muestra al inicio.
    Para mostrar el resto de grupos (si los hay), se debe hacer clic en el desplegable, y se abrirá un listado de los diferentes
    grupos que tiene como hijos.
    \item \textbf{Sección central}.
    La sección central se distribuye en dos pestañas: dispositivos y plantillas.
    Por defecto, se muestra la pestaña de dispositivos.
    Si no se ha seleccionado ningún grupo o el grupo seleccionado no tiene dispositivos, se muestra una indicación de
    que no hay dispositivos en el grupo.
    Si se ha seleccionado un grupo que tiene dispositivos, se muestra una tabla con los dispositivos del grupo y
    diferentes opciones para interactuar con ellos.
    En cada pestaña se muestra la ruta completa del grupo seleccionado, desde el grupo raíz hasta el grupo actual, y
    se puede navegar por estos haciendo clic en el nombre del grupo.
\end{itemize}

\subsubsection{Gestión de grupos}

La gestión de grupos se realiza desde el menú lateral izquierdo, donde se pueden crear, editar y eliminar grupos.
Para ello, se debe hacer clic derecho sobre un grupo y se desplegará un menú con las diferentes opciones:

\imagen{menu_grupos.png}{Menú de gestión de grupos}{0.5}

\begin{itemize}
    \item \textbf{Añadir grupo}.
    Inserta un nuevo grupo.
    Se debe introducir el nombre de este y se generará dentro del grupo seleccionado.
    \item \textbf{Modificar grupo}.
    Edita el nombre del grupo seleccionado.
    \item \textbf{Eliminar grupo}.
    Elimina el grupo seleccionado.
    Si este tiene dispositivos o grupos hijos, se mostrará un mensaje de error indicando que no se puede eliminar.
\end{itemize}

Para la asignación de plantillas a un grupo, hay que seleccionar la pestaña de plantillas.
Si no se selecciona un grupo, la pestaña se mantiene deshabilitada.
Una vez seleccionado un grupo, si no tiene plantillas, se muestra un desplegable con las plantillas disponibles y
los comandos que se ejecutarán al asignar la plantilla.

\imagen{asignar_plantilla_grupo_form.png}{Asignación de plantilla a un grupo}{1}

Si el grupo ya tiene una plantilla asignada, se muestra un botón de ``Eliminar asignación``:

\imagen{desasignar_plantilla_grupo_form.png}{Desasignación de plantilla a un grupo}{1}

\subsubsection{Gestión de dispositivos}

La gestión de dispositivos se realiza desde la pestaña de dispositivos: se muestra una tabla donde se pueden eliminar,
editar y asignar plantillas a los dispositivos:

\imagen{gestion_dispositivos.png}{Tabla de dispositivos}{1}

\imagen{eliminar_dispositivos.png}{Icono para eliminar un dispositivo}{0.1}

\imagen{modificar_dispositivos.png}{Icono para editar un dispositivo}{0.1}

\imagen{asignar_dispositivos.png}{Icono para asignar una plantilla a un dispositivo}{0.1}

\begin{itemize}
    \item \textbf{Eliminar dispositivo}.
    Elimina el dispositivo seleccionado previa confirmación.
    \item \textbf{Añadir dispositivo}.
    Esta funcionalidad se utiliza a modo de ejemplo, ya que para añadir un dispositivo se debe utilizar la aplicación cliente.
    \item \textbf{Editar dispositivo}
    Edita la información del dispositivo seleccionado.
    \item \textbf{Asignar plantilla}.
    Asigna una plantilla al dispositivo seleccionado.
    Se debe seleccionar la plantilla de la lista de plantillas existentes.
    \item \textbf{Desasignar plantilla}.
    Desasigna la plantilla del dispositivo seleccionado.
\end{itemize}

El formulario para editar un dispositivo es el siguiente:

\imagen{modificar_dispositivo_form.png}{Formulario de edición de dispositivo}{1}

Se pueden editar los campos de nombre, sistema operativo y grupo al que pertenece.
Una vez validados los datos, se pulsa en ``Guardar`` y se actualiza la tabla de dispositivos mostrando un mensaje de
éxito.

Para la asignación de plantillas a un dispositivo, al hacer clic en el icono de asignar plantilla, se muestra un
diálogo con una lista de las plantillas disponibles: se selecciona la plantilla deseada y se pulsa en ``Asignar``:

\imagen{asignar_plantilla_dispositivo_form.png}{Asignación de plantilla a un dispositivo}{1}

Una vez seleccionada una plantilla, en la parte inferior del diálogo se muestra un resumen de los comandos que serán
ejecutados en el dispositivo.
Si no aparecen comandos es porque la plantilla no tiene ninguno asociado.

Para desasignar una plantilla, se utiliza el mismo icono de asignar plantilla: ahora se muestra un botón de
``Desasignar plantilla``:

\imagen{desasignar_plantilla_dispositivo_form.png}{Desasignación de plantilla a un dispositivo}{1}

\subsubsection{Gestión de comandos}

Para acceder a la gestión de comandos, se debe pulsar en el enlace de ``Comandos`` que se encuentra en la cabecera
de la plataforma web.

Se muestra una tabla paginada con los comandos existentes y un buscador para filtrarlos por nombre o descripción.
En cada fila de la tabla, se muestra un icono para editar el comando y otro para eliminarlo.
Para añadir un nuevo comando, se debe pulsar en el botón ``Añadir`` que se encuentra en la parte superior derecha de la
página.

\imagen{gestion_comandos.png}{Gestion de comandos}{1}

Tanto para añadir como para editar un comando, se muestra un formulario donde se deben introducir los siguientes campos:
\begin{itemize}
    \item \textbf{Nombre}: nombre del comando.
    \item \textbf{Descripción}: descripción del comando.
    \item \textbf{Comando}: el comando que se pretende ejecutar en el dispositivo.
    Se pueden insertar variables en el comando, las cuales serán sustituidas por los valores que se definan en la
    plantilla a la que se asigne el comando.
    Deben tener el formato \texttt{\{\{nombre\_variable\}\}}, donde \texttt{nombre\_variable} se reemplazará por el
    valor correspondiente.
\end{itemize}

\imagen{añadir_comando.png}{Formulario para un comando}{1}

En este ejemplo, se inserta un comando de prueba que realiza un ping a una dirección IP de manera dinámica.
A la hora de asignar este comando a una plantilla, se puede definir la variable \texttt{direcciónIP} con el valor
deseado, y el comando se ejecutará con ese valor en lugar de la variable.

El botón ''Guardar'' valida los datos introducidos y, si son correctos, añade o edita el comando en la tabla.

Para la eliminación de un comando, se debe pulsar en el icono de ''Eliminar'', y se mostrará un mensaje de
confirmación para eliminar el comando.
Si el comando no está asociado a ninguna plantilla, se eliminará; si lo está, se mostrará un mensaje de error.

\subsubsection{Gestión de plantillas}

La gestión de plantillas se realiza accediendo mediante el enlace de ``Plantillas`` en la cabecera de la plataforma web.
Se comporta de manera similar a la gestión de comandos, con una tabla paginada y un buscador para filtrar las plantillas por
nombre o descripción.
Para añadir una nueva plantilla, se debe pulsar en el botón ``Añadir`` que se encuentra en la parte superior derecha de la
página.
Adicionalmente, la asignación de comandos a la plantilla se realiza desde un nuevo botón que aparece en cada fila
de la tabla.

\imagen{gestion_plantillas.png}{Gestion de plantillas}{1}

El formulario para añadir o editar una plantilla se compone de los siguientes campos:

\begin{itemize}
    \item \textbf{Nombre}: nombre de la plantilla.
    \item \textbf{Descripción}: descripción de la plantilla.
    \item \textbf{Sistema Operativo}: sistema operativo sobre el que se ejecutará la plantilla.
\end{itemize}

\imagen{crear_plantilla.png}{Creación de una plantilla}{1}

Mediante el botón ``Guardar``, los datos son validados y almacenados, y se muestra un mensaje de éxito.

Para eliminar una plantilla, es necesario que no se encuentre asignada a un dispositivo o grupo; si estuviese asignada,
se mostrará un mensaje de error.

La asignación de comandos a una plantilla existente se realiza en una página aislada, de manera que el usuario se
pueda enfocar únicamente en la tarea.
Este botón se encuentra disponible en cada registro de la tabla de plantillas.

\imagen{icono_asignar_comandos.png}{Icono para asignar comandos a una plantilla}{0.1}

La asignación de comandos consta de dos columnas:
\begin{itemize}
    \item \textbf{Comandos disponibles}.
    Aquí se muestran todos los comandos que pueden aplicarse a la plantilla.
    Se pueden realizar búsquedas tal y como se hace para los comandos y plantillas.
    Para añadir un comando y asignarlo a la plantilla, basta con pulsar sobre el botón que tiene cada comando con el
    símbolo ''+'' y automáticamente se desplaza a la siguiente columna.
    \item \textbf{Comandos asignados}.
    Aquellos comandos que se encontraban previamente asignados aparecerán en esta columna.
    Cuando se añaden desde la anterior columna, se colocarán al final.
    No obstante, mediante las flechas ↑ y ↓ puede modificarse el orden en el que se ejecutarán los comandos, se ejecutan
    de arriba hacia abajo.
    Por último, se puede eliminar un comando pulsando sobre el botón que aparece al lado de las flechas.

\end{itemize}

\imagen{asignar_comandos.png}{Asignar comandos a una plantilla}{1}

\imagen{asignar_comandos_2.png}{Comandos asignado a una plantilla}{1}

En aquellos comandos que posean variables dinámicas, deberá añadirse obligatoriamente un valor que sustituya a la
variable; aparecerá además el comando resultante.

Cuando hayamos realizado los cambios necesarios, pulsamos sobre ``Guardar`` para aplicar los cambios.

\subsection{Aplicación cliente}

Al iniciar la aplicación (con permisos de administrador) por primera vez, cuando dispositivo aún no se encuentra
registrado, se mostrará una ventana para iniciar sesión y un recuadro que muestra los logs de la aplicación en tiempo
real.

\imagen{app_start.png}{Inicio de la aplicación cliente}{1}

Cada vez que se inicia la aplicación, se restaura la configuración base y se aplica la configuración más actual para el
dispositivo, siempre y cuando esté registrado.

\subsubsection{Dar de alta a un dispositivo}

Para dar de alta un dispositivo es necesario iniciar sesión, de forma que se habilitan los botones para iniciar el
proceso.
Si las credenciales son correctas, se mostrará un mensaje y se carga el panel de usuario.

\imagen{app_login.png}{Mensaje de inicio de sesión}{0.2}

\imagen{app_user.png}{Panel de usuario}{1}

Cuando el equipo no se encuentre registrado, se muestra el botón ``Registrar equipo``.
Una vez realizado el registro (si se ha realizado de manera satisfactoria), se informa al usuario de la huella con
la que ha sido dado de alta.

Tras realizar el alta, se realiza una copia de seguridad de las configuraciones de firewall y de red del dispositivo
para ser restaurada al eliminar el equipo o tras una nueva aplicación de una plantilla.

\imagen{app_register.png}{Mensaje de alta del equipo}{1}

De forma paralela, la aplicación comprobará periódicamente si hay configuraciones que aplicar.
Este registro queda reflejado en el recuadro de logs de la aplicación, que siempre se mantiene activo.

Para comprobar que se ha registrado correctamente, accedemos a la plataforma web, y en el grupo raíz ``UBU`` debería
aparecer el dispositivo:

\imagen{dispositivo_registrado.png}{Dispositivo dado de alta desde el cliente}{1}

\subsubsection{Asignar plantilla a un dispositivo}

Seleccionamos el dispositivo registrado, pulsamos en el botón de asignación de plantilla y escogemos la que
queremos aplicar:

\imagen{asignar_plantilla_equipo.png}{Asignación de una plantilla al dispositivo registrado}{1}

Mediante la plantilla que se ha escogido, deshabilitaremos el firewall del dispositivo.
Hecho esto, el cliente se encargará de aplicar la configuración al equipo.

La aplicación registra la ejecución de comandos y lista aquellos que se han ejecutado:

\imagen{comandos_ejecutados.png}{Registro de un comando ejecutado satisfactoriamente}{1}

Además, podemos comprobar en el equipo que el firewall se encuentra desactivado:

\imagen{firewall_off.png}{Firewall desactivado}{1}

Al aplicar la plantilla, el dispositivo registra el instante en el que fue aplicada, de forma que, mientras no se actualize la plantilla
en la aplicación web, no se aplicará una nueva configuración.
Del mismo modo, si asignamos una nueva plantilla al dispositivo, se ejecutarán los comandos de esta última.
Ante cualquier cambio de comandos y plantillas, se restaura la copia de seguridad que se realizó previamente en el
registro.

\subsubsection{Dar de baja un dispositivo}

El proceso de baja es similar al de alta: requiere iniciar sesión y que el dispositivo se encuentre registrado.
Si se cumplen estas condiciones, podemos pulsar en el botón ``Eliminar dispositivo`` y confirmar la eliminación:

\imagen{app_delete.png}{Mensaje de baja del equipo}{1}

De esta manera, se borran todos los registros del equipo, tanto en el propio dispositivo como en la base de datos, y
se restaura la configuración inicial.
Como la plantilla que deshabilitaba el firewall se elimina, el firewall vuelve a estar activo.

\subsubsection{Consideraciones a tener en cuenta}

Existen comandos que deshabilitan el acceso a Internet en su totalidad, por lo que el acceso a la API se vuelve imposible.
Para volver a tener acceso, al iniciar la aplicación nuevamente, se restaura la configuración de red y vuelve
a comprobar qué plantilla tiene asignada: si la plantilla asignada sigue deshabilitando la conexión, volverá a aplicarse
nuevamente.

Aunque se deshabilite el tráfico entrante y saliente, la conexión a la API estará siempre habilitada para
que el dispositivo pueda seguir actualizando su configuración.