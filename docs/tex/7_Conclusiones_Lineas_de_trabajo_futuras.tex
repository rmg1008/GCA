\capitulo{7}{Conclusiones y Líneas de trabajo futuras}

Una vez finalizado y revisado el contenido del proyecto, se pueden extraer diversas conclusiones.

En lo que al ámbito personal respecta, me ha resultado útil para aprender a gestionar mejor la carga de
trabajo; he sido consciente de que no es eficiente tratar de tener todo hecho de la manera más rápida posible,
sino que, si nos paramos a realizar un análisis previo de todo aquello que se necesita hacer y lo organizamos en
prioridades, podremos dedicar tiempo a aquellas cosas que sean más necesarias y nos permitan abrirnos paso a completar
otras tareas que, a priori, antes de este análisis parecían imposibles.
El trabajo diario y la estructuración previa de tareas resulta mucho más eficiente que dedicar un sinfín de horas al
proyecto sin un orden previamente establecido, ya que la falta de organización puede conllevar a un uso
improductivo del tiempo dedicado.
Al tener bien distribuidas las tareas, cuando me encontraba atascado con alguna de ellas, traté de continuar con otra
y volver a la primera en otro momento; así conseguía ``resetear`` y regresar con un punto de vista diferente que,
generalmente, me permitía resolver el problema.

El resultado del proyecto, si bien no cuenta con un número extenso de funcionalidades, incluye todos los elementos
necesarios para realizar una gestión de los dispositivos en las aulas de una manera sencilla, pero eficaz.
La interfaz web se caracteriza por su facilidad de uso; es intuitiva, no precisa de un manual extenso para su utilización porque los
textos, iconos e imágenes permiten diferenciar las distintas funcionalidades.
La aplicación cliente, siguiendo la misma línea, implementa lo básico para posibilitar la gestión de los dispositivos
y mostrar en todo momento qué está sucediendo.
La API es más compleja, ya que requiere conocimientos en Java y Spring Boot para poder entender toda su funcionalidad.

Teniendo en cuenta todo esto, este proyecto es una primera versión de lo que puede llegar a ser.
Soy consciente de que se trata de un proyecto con una gran capacidad de mejora: al contar con permisos de administrador
en la ejecución de la aplicación, abre un abanico de infinitas posibilidades a nuevas funcionalidades.
Además, la propia API ha sido construida para ser escalable y poder ejecutarse en cualquier tipo de plataforma y de
sistema operativo.

Dentro de la columna de ``Backlog`` están aquellas tareas que, debido al tiempo disponible para la realización del TFG,
no han podido ser completadas, las cuales son ligeras mejoras del proyecto de cara al futuro.

Por último, considero importante plantear una serie de ideas que pueden ser realizadas en versiones más avanzadas de GCA:
\begin{itemize}
    \item La creación de una aplicación cliente para sistemas operativos Linux y MacOS.
    \item Almacenar en la nube las copias de seguridad realizadas en los dispositivos y llevar un histórico.
    \item Implementar un panel \textit{Administrador} para realizar la gestión de otros usuarios.
    \item Generar informes para estudiar qué comandos se utilizan más y en qué orden.
    \item Actualizar el ``Look and Feel`` de la aplicación cliente para Windows.
    \item Utilizar SignTool en Windows para firmar la app cliente y que sea categorizada como segura.
    \item Desarrollar un diseñador visual para implementar workflows con plantillas y comandos.
    \item Machine Learning para analizar tendencias de tráfico, detectar patrones inusuales y recomendar ajustes.
\end{itemize}