\apendice{Plan de Proyecto Software}

\section{Introducción}

La fase de planificación constituye un elemento clave e imprescindible en cualquier proyecto.
Mediante una planificación adecuada, se incrementan notablemente las posibilidades de éxito, ya
que permite anticipar problemas, distribuir adecuadamente los recursos y organizar las tareas de una
forma eficiente.

Por este motivo, en este anexo se detallan aquellos aspectos necesarios para el desarrollo
del proyecto.

Dividiremos el plan en dos apartados:
\begin{itemize}
    \item \textbf{La planificación temporal:} donde se establece el modo de organizar las actividades a realizar,
    una estimación del tiempo que se va a dedicar a cada una de ellas y las fechas de inicio y fin del proyecto.
    \item \textbf{El estudio de viabilidad:} en el que se analiza si el proyecto es factible desde el punto de vista
financiero y legal, atendiendo a los elementos necesarios para su realización.
\end{itemize}

\section{Planificación temporal}

El proyecto se ha realizado bajo un marco ágil de gestión de proyectos, basado en Scrum, aunque ha sido adaptado a las
características específicas del contexto.


Debido a la inexistencia de un equipo, la falta de un entorno estructurado (roles definidos, reuniones diarias, retrospectivas
y \textit{demos}), esta planificación no se ha podido aplicar en su totalidad; aún así, se han mantenido varios de sus
fundamentos claves, aunque con ciertos matices:

\begin{itemize}
    \item \textbf{Iteraciones cortas:} organizadas en sprints de 2 semanas, durante los cuales se han desarrollado
    nuevas funcionalidades de manera incremental.
    \item \textbf{Equipos autoorganizados:} aunque no exista un equipo como tal, el desarrollo se ha producido de una manera autónoma.
    \item \textbf{Reuniones semanales:} se programó una reunión formal todos los lunes para revisar el estado del proyecto y
    resolver posibles dudas surgidas durante la semana anterior.
    \item \textbf{Backlog del producto:} en el que se van añadiendo de forma continua todas las tareas pendientes de realizar (ya sean de
    desarrollo, documentación\ldots) y se mantiene un listado de tareas ordenadas por prioridad.
    \item \textbf{Planificación de los sprints:} en la que se estudian las tareas que se realizarán para entregar en el
    siguiente incremento, atendiendo a características como prioridad y complejidad.
\end{itemize}

A cada tarea se le asigna un valor estimado de esfuerzo, denominado \textit{story point}, los cuales están basados en la secuencia de
Fibonacci (1, 2, 3, 5, 8\ldots).

Un \textit{story point} representa una combinación de dificultad y tiempo estimado; en este sentido, decimos que una
tarea de 3 puntos comparada con una de 5 se considera más sencilla, y, por tanto, se realizará en menor tiempo.
Cabe mencionar que este sistema permite hacer estimaciones de tiempo para las tareas, pero no establece una equivalencia exacta con días de trabajo.

Esta forma de trabajar no es la más exacta, pero permite flexibilizar el trabajo, lo que resulta útil en este contexto
específico en el que la capacidad disponible de trabajo puede variar en el tiempo: incluso en entornos ideales, los
equipos pueden verse afectados por situaciones como bajas, vacaciones\ldots Por eso, la planificación debe poder ser adaptable.

\textbf{Sprint 0 (25/02/2025 - 05/03/2025)}
Este primer sprint sirvió como punto de partida: se estableció una reunión inicial con el tutor, \textbf{Pedro Renedo}, en la que
abordamos y concretamos el tema central del proyecto que se pretende realizar.

Se creó el repositorio inicial en GitHub y estudiamos las posibles herramientas y lenguajes de programación que se utilizarían
en el desarrollo.

Además, se estimaron  \textbf{7 \textit{story points}}, que correspondieron aproximadamente con la semana que duró el sprint.
Esta duración fue una excepción, ya que se trató del primero; después se optó por alargar este tiempo a dos
semanas, ya que observamos que un sprint de solo siete días era demasiado corto cuando se introducían tareas de desarrollo.

\textbf{Sprint 1 (05/03/2025 - 20/03/2025)}

Durante este sprint se profundizó en el análisis del proyecto, tratando de responder a preguntas clave como:
\textbf{¿Qué se necesita realizar?¿Cómo ha de hacerse?¿Con base a qué surge esta necesidad? \ldots}

Además, se estudió el tipo de aplicación que se iba a necesitar en los dispositivos, se modeló una primera versión de
base de datos y se creó la estructura inicial de la aplicación backend en Java.
En este punto surgió un problema técnico: los ordenadores del centro son clonados, por lo que tienen el mismo GUID y no sirve como
identificador único, por lo que hay que se hizo necesario seguir investigando para encontrar algún método alternativo.

Se estimaron \textbf{19 \textit{story points}} de trabajo para los 15 días de trabajo.


\textbf{Sprint 2 (20/03/2025 - 02/04/2025)}

Continuamos con la línea de definición de requisitos y objetivos funcionales, se desarrollaron los CU principales, y además, se
diseñó un método de autentificación para los equipos, el cual combina el GUID con el nombre de los equipos, aprovechando
que estos sí son únicos para cada uno.

Por otro lado, se creó la estructura de la aplicación frontal.

Se estimaron \textbf{15.5 \textit{story points}} para los 15 días de trabajo; en este caso, se ajustó a 1 \textit{story point} por día prácticamente.


\textbf{Sprint 3 (02/04/2025 - 16/04/2025)}

En esta etapa comenzaron las tareas de desarrollo del backend.
Para ello, se creó una base de datos inicial con MariaDB después de refinar el diagrama E-R actual, y se implementaron
las bases de nuestra API con Spring Boot, configurando los primeros end points e implementando medidas de seguridad
en las llamadas mediante login y generación de tokens.

Se estimaron \textbf{21 \textit{story points}} en este sprint, ya que la disponibilidad prevista era superior a la de otros sprints.

\textbf{Sprint 4 (16/04/2025 - 30/04/2025)}

Avanzamos creando nuevos endpoints básicos, sentando las bases que utilizaremos en la aplicación web para gestionar
los diferentes grupos y dispositivos.
En paralelo, trabajamos en el desarrollo de nueva documentación, definiendo los requisitos no funcionales y las técnicas y
herramientas que se pretenden utilizar, así como el estudio de viabilidad del proyecto.

Es estimaron \textbf{17 \textit{story points}} para este sprint, ya que se consideró que la disponibilidad de tiempo
iba a ser menor que en el anterior.

\textbf{Sprint 5 (30/04/2025 - 14/05/2025)}

Continuamos con el desarrollo de la API, pero en este caso, también diseñamos una posible aproximación a la interfaz
de usuario, generamos diagramas de los casos de uso más importantes, y comenzamos a implementar la parte de la aplicación
frontal utilizando Angular.

En la reunión semanal con el tutor, se aclararon puntos relevantes relativos a la aplicación: uno de ellos fue la
posibilidad de generar comandos con valores dinámicos, de forma que, en el futuro, para cada plantilla, se puedan asignar
valores diferentes en función de la plantilla utilizada.

Otro punto relevante que surgió fue la necesidad de investigar qué hacer con los equipos a los que se les deshabilita
el acceso a la red, ya que no se podrían gestionar de manera remota.
Una posibilidad que se barajó fue realizar una copia del estado del equipo antes de aplicar la plantilla, para que cuando se
reinicie, se restaure el estado anterior, de manera que el equipo se mantenga tal y como estaba antes de aplicar la plantilla.

Se estimaron \textbf{20 \textit{story points}} para este sprint, ya que se consideró que la disponibilidad iba a ser mayor y en
este punto del proyecto, las bases ya estaban más asentadas, el trabajo se hacía más fluido y había menos dudas.

\textbf{Sprint 6 (14/05/2025 - 28/05/2025)}

Este Sprint es uno de los más críticos: se actualizó la aplicación cliente, permitiendo obtener y aplicar las plantillas
de manera remota.
De forma paralela, completamos la interfaz web, permitiendo gestionar comandos y plantillas de forma visual.
En cuanto a test se refiere, mejoramos la cobertura del código, añadiendo pruebas unitarias y de integración.

Se estimaron \textbf{20 \textit{story points}} para este sprint porque disponíamos de más tiempo para la realización de las
tareas.

\textbf{Sprint 7 (28/05/2025 - 04/06/2025)}

El penúltimo sprint antes de la entrega final consistió en rellenar la documentación del proyecto, así como añadir
logs en el backend para mejorar la trazabilidad de las acciones realizadas.
Tras realizar una demostración del proyecto al tutor, se crearon nuevas tareas para mejorar la usabilidad de la
aplicación cliente.

La duración de este sprint y el siguiente se redujeron a una semana, ya que al tratarse de la fase final, es más sencillo
contar con períodos más cortos de tiempo y poder así ajustar si surgen problemas inesperados.
Se estimaron \textbf{10 \textit{story points}} para este sprint.

\textbf{Sprint 8 (04/06/2025 - 11/06/2025)}

\section{Estudio de viabilidad}

\subsection{Viabilidad económica}

En este apartado analizaremos los costes y beneficios de haber implementado el proyecto con una intención real de llevarlo
a cabo.

\textbf{Costes}

\textbf{Costes de personal:} todo el desarrollo ha sido realizado por un único desarrollador en un tiempo total de cinco
meses comprendidos entre febrero y junio.
El sueldo base promedio para un puesto de desarrollador ``Full Stack Developer`` en España es de 30 mil €/año brutos, de acuerdo con
la información publicada en Glassdor\cite{glassdoorSueldo}


Calculamos el salario en base a una persona de 30 años, soltera y sin hijos\cite{seguridadSocial2025}

\begin{table}[H]
    \centering
    \begin{tabular}{| c | c |}
        \Xhline{2\arrayrulewidth}
        \textbf{Concepto} & \textbf{Coste} \\ \Xhline{2\arrayrulewidth}
        Salario neto & 1929,8€ \\ \hline
        Retención IRPF & 411,5€ \\ \hline
        Cuota seguridad social & 148,75€ \\ \hline
        Salario mensual bruto & 2500€ \\ \hline
        \Xhline{2\arrayrulewidth}
        \textbf{Total 5 meses} & \textbf{12500€} \\ \Xhline{2\arrayrulewidth}
    \end{tabular}
    \caption{\textit{Nómina del empleado.}}
\end{table}

\begin{table}[H]
    \centering
    \begin{tabular}{| c | c |}
        \Xhline{2\arrayrulewidth}
        \textbf{Concepto} & \textbf{Coste} \\ \Xhline{2\arrayrulewidth}
        Salario bruto & 2500€ \\ \hline
        Contingencias Comunes & 590€ \\ \hline
        Accidentes de Trabajo y Enfermedades Profesionales (AT y EP) & 22,5€ \\ \hline
        Desempleo & 137,5€ \\ \hline
        Formación Profesional & 15€ \\ \hline
        Fondo de garantía salarial (FOGASA) & 5€ \\ \hline
        \Xhline{2\arrayrulewidth}
        \textbf{Total 5 meses} & \textbf{16350€} \\ \Xhline{2\arrayrulewidth}
    \end{tabular}
    \caption{\textit{Coste para la empresa.}}
\end{table}

\textbf{Costes de hardware}

Para el hardware utilizado, no se emplearon demasiadas herramientas más allá de un ordenador para la realización del proyecto,
que se hará a medida, a priori, con un dispositivo de 16GB de RAM, SSD NVMe, procesador actual y una tarjeta gráfica estándar
será suficiente.
En el precio viene ya incluido una licencia de Win 11.

\begin{gather*}
    \text{Coste amortizado} = \frac{\text{Coste total del equipo}}{\text{Vida útil (en meses)}} \times \text{Meses del proyecto}\\
    \text{Coste amortizado} = \frac{1200\,€}{48\,\text{meses}} \times 5\,\text{meses} = 125\,€\\
\end{gather*}

\begin{table}[H]
    \centering
    \begin{tabular}{| c | c |}
        \Xhline{2\arrayrulewidth}
        \textbf{Concepto} & \textbf{Coste} \\ \Xhline{2\arrayrulewidth}
        Ordenador a medida & 1200€ \\ \hline
        Amortización durante 5 meses & 125€ \\ \hline
        \Xhline{2\arrayrulewidth}
        \textbf{Total amortizado} & \textbf{125€} \\ \Xhline{2\arrayrulewidth}
    \end{tabular}
    \caption{\textit{Costes del hardware.}}
\end{table}
\textbf{Costes de software}

En este apartado tendremos en cuenta las licencias necesarias para el desarrollo del proyecto: para los IDE's existen
dos opciones, o bien utilizar las licencias gratuitas ``Community Edition``, o las de pago\cite{jetbrainsPrecios}
Escogemos esta última porque cuenta con características que ayudan a desarrollar rápidamente nuestro proyecto.

\begin{table}[H]
    \centering
    \begin{tabular}{| c | c | c |}\hline
        \Xhline{2\arrayrulewidth}
        \textbf{Concepto} & \textbf{Coste} \\ \Xhline{2\arrayrulewidth}
        All Products Pack (mes) & 94,26€ \\ \hline
        Visual Studio Code & 0€ \\ \hline
        \Xhline{2\arrayrulewidth}
        \textbf{Total 5 meses} & \textbf{471,3€} \\ \Xhline{2\arrayrulewidth}
    \end{tabular}
    \caption{\textit{Costes del software.}}
\end{table}

\textbf{Costes variados}

En este apartado incluimos otros costes a tener en cuenta:


\begin{table}[H]
    \centering
    \begin{tabular}{| c | c |}\hline
        \Xhline{2\arrayrulewidth}
        \textbf{Concepto} & \textbf{Coste} \\ \Xhline{2\arrayrulewidth}
        Dominio web & 20€ \\ \hline
        Hosting / contendor & 50€\\ \hline
        Internet & 100€ \\ \hline
        \Xhline{2\arrayrulewidth}
        \textbf{Total} & \textbf{170€} \\ \Xhline{2\arrayrulewidth}
    \end{tabular}
    \caption{\textit{Costes varios.}}
\end{table}

\textbf{Costes totales}

Se muestra el sumatorio de todos los costes:

\begin{table}[H]
    \centering
    \begin{tabular}{| c | c |}\hline
        \Xhline{2\arrayrulewidth}
        \textbf{Concepto} & \textbf{Coste} \\ \Xhline{2\arrayrulewidth}
        Personal & 16350€ \\ \hline
        Hardware & 125€\\ \hline
        Software & 471,3€ \\ \hline
        Variados & 170€ \\ \hline
        \Xhline{2\arrayrulewidth}
        \textbf{Total} & \textbf{17116,3€} \\ \Xhline{2\arrayrulewidth}
    \end{tabular}
    \caption{\textit{Costes totales del proyecto.}}
\end{table}

\textbf{Beneficios}

Se trata de un proyecto con fines estrictamente académicos, pues la idea es que pueda llegar a ser desarrollado para
la Universidad de Burgos.

No obstante, se podría considerar establecer algún tipo de suscripción (ya sea mensual o anual) y que pueda ser
utilizado por otras entidades educativas o, incluso, por empresas privadas.

\begin{table}[H]
    \centering
    \begin{tabular}{| c | c | c | c | }\hline
        \Xhline{2\arrayrulewidth}
        \textbf{Tipo} & \textbf{Objetivo} & \textbf{Precio mensual} & \textbf{Precio anual} \\ \Xhline{2\arrayrulewidth}
        Básica & Centros educativos pequeños & 49,90€ & 499€ \\ \hline
        Profesional & Instituciones grandes & 89,90€ & 899€ \\ \hline
        Empresarial & Empresas privadas & 149,90€ & 1149€ \\ \hline
        Premium & Incluye personalización total & 250€ & 2500€ \\ \hline
    \end{tabular}
    \caption{\textit{Monetización del proyecto.}}
\end{table}

Para calcular el número de suscripciones que necesitaríamos para cubrir gastos, utilizamos la siguiente fórmula:

\[
    \text{Suscripciones necesarias} = \frac{17.116{,}3\,€}{\text{Precio de la suscripción anual}}
\]

\begin{table}[H]
    \centering
    \begin{tabular}{| c | c | c | }\hline
        \Xhline{2\arrayrulewidth}
        \textbf{Tipo} & \textbf{Precio anual} & \textbf{Suscripciones necesarias} \\ \Xhline{2\arrayrulewidth}
        Básica & 499€ & 35 \\ \hline
        Profesional & 899€ & 19 \\ \hline
        Empresarial & 1149€ & 15 \\ \hline
        Premium & 2500€ & 7 \\ \hline
    \end{tabular}
    \caption{\textit{Suscripciones necesarias para cubrir gastos.}}
\end{table}

De esta manera, calculamos el número de suscripciones necesarias (si solo se tuviesen de un mismo tipo)
para llegar a cubrir los gastos totales del proyecto.
También se podría utilizar un modelo mixto e incluir suscripciones de cada tipo.

\subsection{Viabilidad legal}

Abordamos aquellos temas relacionados con el contexto legal del proyecto, así como lo relativo a las licencias.

La implementación del proyecto ha sido propia, a excepción de las librerías y dependencias de terceros.

En este punto, diferenciaremos las dependencias utilizadas en el cliente (Python) de las utilizadas en la API (Java) y
la interfaz de gestión de usuario (Angular)

\subsubsection{Python}

\begin{longtable}[]{@{}llll@{}}
    \toprule
    \textbf{Dependencia} & \textbf{Versión} & \textbf{Descripción} & \textbf{Licencia} \\
    \midrule
    \endfirsthead

    \multicolumn{4}{c}%
    {\tablename\ \thetable{} -- \textit{continuación}} \\
    \toprule
    \textbf{Dependencia} & \textbf{Versión} & \textbf{Descripción} & \textbf{Licencia} \\
    \midrule
    \endhead

    \begin{minipage}[t]{0.18\columnwidth}\raggedright\strut altgraph \strut \end{minipage} & \begin{minipage}[t]{0.10\columnwidth}\raggedright\strut 0.17.4 \strut \end{minipage} & \begin{minipage}[t]{0.49\columnwidth}\raggedright\strut Análisis y representación de grafos de objetos de Python. \strut \end{minipage} & \begin{minipage}[t]{0.11\columnwidth}\raggedright\strut MIT \strut \end{minipage} \tabularnewline

    \begin{minipage}[t]{0.18\columnwidth}\raggedright\strut certifi \strut \end{minipage} & \begin{minipage}[t]{0.10\columnwidth}\raggedright\strut 2025.1.31 \strut \end{minipage} & \begin{minipage}[t]{0.49\columnwidth}\raggedright\strut Certificados raíz para validar la seguridad SSL. \strut \end{minipage} & \begin{minipage}[t]{0.11\columnwidth}\raggedright\strut MPL-2.0 \strut \end{minipage} \tabularnewline

    \begin{minipage}[t]{0.18\columnwidth}\raggedright\strut charset-normalizer \strut \end{minipage} & \begin{minipage}[t]{0.10\columnwidth}\raggedright\strut 3.4.1 \strut \end{minipage} & \begin{minipage}[t]{0.49\columnwidth}\raggedright\strut Detección de codificación de texto similar a chardet. \strut \end{minipage} & \begin{minipage}[t]{0.11\columnwidth}\raggedright\strut MIT \strut \end{minipage} \tabularnewline

    \begin{minipage}[t]{0.18\columnwidth}\raggedright\strut idna \strut \end{minipage} & \begin{minipage}[t]{0.10\columnwidth}\raggedright\strut 3.10 \strut \end{minipage} & \begin{minipage}[t]{0.49\columnwidth}\raggedright\strut Soporte de nombres de dominio internacionalizados (IDNA). \strut \end{minipage} & \begin{minipage}[t]{0.11\columnwidth}\raggedright\strut BSD \strut \end{minipage} \tabularnewline

    \begin{minipage}[t]{0.18\columnwidth}\raggedright\strut packaging \strut \end{minipage} & \begin{minipage}[t]{0.10\columnwidth}\raggedright\strut 24.2 \strut \end{minipage} & \begin{minipage}[t]{0.49\columnwidth}\raggedright\strut Utilidades para manejo de versiones y requerimientos. \strut \end{minipage} & \begin{minipage}[t]{0.11\columnwidth}\raggedright\strut Apache-2.0 \strut \end{minipage} \tabularnewline

    \begin{minipage}[t]{0.18\columnwidth}\raggedright\strut pefile \strut \end{minipage} & \begin{minipage}[t]{0.10\columnwidth}\raggedright\strut 2023.2.7 \strut \end{minipage} & \begin{minipage}[t]{0.49\columnwidth}\raggedright\strut Lectura y modificación de archivos PE (Portable Executable). \strut \end{minipage} & \begin{minipage}[t]{0.11\columnwidth}\raggedright\strut MIT \strut \end{minipage} \tabularnewline

    \begin{minipage}[t]{0.18\columnwidth}\raggedright\strut pillow \strut \end{minipage} & \begin{minipage}[t]{0.10\columnwidth}\raggedright\strut 11.1.0 \strut \end{minipage} & \begin{minipage}[t]{0.49\columnwidth}\raggedright\strut Biblioteca de procesamiento de imágenes (fork de PIL). \strut \end{minipage} & \begin{minipage}[t]{0.11\columnwidth}\raggedright\strut HPND \strut \end{minipage} \tabularnewline

    \begin{minipage}[t]{0.18\columnwidth}\raggedright\strut pip \strut \end{minipage} & \begin{minipage}[t]{0.10\columnwidth}\raggedright\strut 25.0.1 \strut \end{minipage} & \begin{minipage}[t]{0.49\columnwidth}\raggedright\strut Instalador oficial de paquetes Python. \strut \end{minipage} & \begin{minipage}[t]{0.11\columnwidth}\raggedright\strut MIT \strut \end{minipage} \tabularnewline

    \begin{minipage}[t]{0.18\columnwidth}\raggedright\strut pyinstaller \strut \end{minipage} & \begin{minipage}[t]{0.10\columnwidth}\raggedright\strut 6.12.0 \strut \end{minipage} & \begin{minipage}[t]{0.49\columnwidth}\raggedright\strut Empaquetado de apps Python en ejecutables. \strut \end{minipage} & \begin{minipage}[t]{0.11\columnwidth}\raggedright\strut GPL-2.0 \strut \end{minipage} \tabularnewline

    \begin{minipage}[t]{0.18\columnwidth}\raggedright\strut pyinstaller-hooks-contrib \strut \end{minipage} & \begin{minipage}[t]{0.10\columnwidth}\raggedright\strut 2025.2 \strut \end{minipage} & \begin{minipage}[t]{0.49\columnwidth}\raggedright\strut Hooks adicionales para PyInstaller. \strut \end{minipage} & \begin{minipage}[t]{0.11\columnwidth}\raggedright\strut GPL-2.0 \strut \end{minipage} \tabularnewline

    \begin{minipage}[t]{0.18\columnwidth}\raggedright\strut pystray \strut \end{minipage} & \begin{minipage}[t]{0.10\columnwidth}\raggedright\strut 0.19.5 \strut \end{minipage} & \begin{minipage}[t]{0.49\columnwidth}\raggedright\strut Gestión de iconos en la bandeja del sistema. \strut \end{minipage} & \begin{minipage}[t]{0.11\columnwidth}\raggedright\strut MIT \strut \end{minipage} \tabularnewline

    \begin{minipage}[t]{0.18\columnwidth}\raggedright\strut pywin32-ctypes \strut \end{minipage} & \begin{minipage}[t]{0.10\columnwidth}\raggedright\strut 0.2.3 \strut \end{minipage} & \begin{minipage}[t]{0.49\columnwidth}\raggedright\strut Acceso a funcionalidades Win32 usando ctypes. \strut \end{minipage} & \begin{minipage}[t]{0.11\columnwidth}\raggedright\strut MIT \strut \end{minipage} \tabularnewline

    \begin{minipage}[t]{0.18\columnwidth}\raggedright\strut requests \strut \end{minipage} & \begin{minipage}[t]{0.10\columnwidth}\raggedright\strut 2.32.3 \strut \end{minipage} & \begin{minipage}[t]{0.49\columnwidth}\raggedright\strut Biblioteca HTTP simple y elegante para Python. \strut \end{minipage} & \begin{minipage}[t]{0.11\columnwidth}\raggedright\strut Apache-2.0 \strut \end{minipage} \tabularnewline

    \begin{minipage}[t]{0.18\columnwidth}\raggedright\strut setuptools \strut \end{minipage} & \begin{minipage}[t]{0.10\columnwidth}\raggedright\strut 78.1.0 \strut \end{minipage} & \begin{minipage}[t]{0.49\columnwidth}\raggedright\strut Herramienta de empaquetado y distribución. \strut \end{minipage} & \begin{minipage}[t]{0.11\columnwidth}\raggedright\strut MIT \strut \end{minipage} \tabularnewline

    \begin{minipage}[t]{0.18\columnwidth}\raggedright\strut six \strut \end{minipage} & \begin{minipage}[t]{0.10\columnwidth}\raggedright\strut 1.17.0 \strut \end{minipage} & \begin{minipage}[t]{0.49\columnwidth}\raggedright\strut Compatibilidad entre Python 2 y 3. \strut \end{minipage} & \begin{minipage}[t]{0.11\columnwidth}\raggedright\strut MIT \strut \end{minipage} \tabularnewline

    \begin{minipage}[t]{0.18\columnwidth}\raggedright\strut urllib3 \strut \end{minipage} & \begin{minipage}[t]{0.10\columnwidth}\raggedright\strut 2.3.0 \strut \end{minipage} & \begin{minipage}[t]{0.49\columnwidth}\raggedright\strut Cliente HTTP con soporte para conexión persistente. \strut \end{minipage} & \begin{minipage}[t]{0.11\columnwidth}\raggedright\strut MIT \strut \end{minipage} \tabularnewline

    \bottomrule
    \caption{Dependencias del proyecto Python}
\end{longtable}


\subsubsection{Java}

\begin{longtable}[]{@{}p{0.18\columnwidth}p{0.10\columnwidth}p{0.49\columnwidth}p{0.11\columnwidth}@{}}
    \toprule
    \textbf{Dependencia} & \textbf{Versión} & \textbf{Descripción} & \textbf{Licencia} \\
    \midrule
    \endfirsthead

    \multicolumn{4}{c}%
    {\tablename\ \thetable{} -- \textit{continuación}} \\
    \toprule
    \textbf{Dependencia} & \textbf{Versión} & \textbf{Descripción} & \textbf{Licencia} \\
    \midrule
    \endhead

    \begin{minipage}[t]{0.18\columnwidth}\raggedright spring-boot-starter \end{minipage} & \begin{minipage}[t]{0.10\columnwidth}\raggedright 3.4.3 \end{minipage} & \begin{minipage}[t]{0.49\columnwidth}\raggedright Starter básico para aplicaciones Spring Boot. \end{minipage} & \begin{minipage}[t]{0.11\columnwidth}\raggedright Apache-2.0 \end{minipage} \\ \tabularnewline

    \begin{minipage}[t]{0.18\columnwidth}\raggedright spring-boot-starter-web \end{minipage} & \begin{minipage}[t]{0.10\columnwidth}\raggedright 3.4.3 \end{minipage} & \begin{minipage}[t]{0.49\columnwidth}\raggedright Starter para crear aplicaciones web con Spring MVC. \end{minipage} & \begin{minipage}[t]{0.11\columnwidth}\raggedright Apache-2.0 \end{minipage} \\ \tabularnewline

    \begin{minipage}[t]{0.18\columnwidth}\raggedright spring-boot-starter-actuator \end{minipage} & \begin{minipage}[t]{0.10\columnwidth}\raggedright 3.4.3 \end{minipage} & \begin{minipage}[t]{0.49\columnwidth}\raggedright Starter para monitorización y gestión de aplicaciones. \end{minipage} & \begin{minipage}[t]{0.11\columnwidth}\raggedright Apache-2.0 \end{minipage} \\ \tabularnewline

    \begin{minipage}[t]{0.18\columnwidth}\raggedright spring-boot-starter-data-jpa \end{minipage} & \begin{minipage}[t]{0.10\columnwidth}\raggedright 3.4.3 \end{minipage} & \begin{minipage}[t]{0.49\columnwidth}\raggedright Starter para usar Spring Data JPA con Hibernate. \end{minipage} & \begin{minipage}[t]{0.11\columnwidth}\raggedright Apache-2.0 \end{minipage} \\ \tabularnewline

    \begin{minipage}[t]{0.18\columnwidth}\raggedright spring-boot-starter-security \end{minipage} & \begin{minipage}[t]{0.10\columnwidth}\raggedright 3.4.3 \end{minipage} & \begin{minipage}[t]{0.49\columnwidth}\raggedright Starter para habilitar seguridad con Spring Security. \end{minipage} & \begin{minipage}[t]{0.11\columnwidth}\raggedright Apache-2.0 \end{minipage} \\ \tabularnewline

    \begin{minipage}[t]{0.18\columnwidth}\raggedright spring-boot-starter-validation \end{minipage} & \begin{minipage}[t]{0.10\columnwidth}\raggedright 3.4.3 \end{minipage} & \begin{minipage}[t]{0.49\columnwidth}\raggedright Starter para validación de datos usando Bean Validation. \end{minipage} & \begin{minipage}[t]{0.11\columnwidth}\raggedright Apache-2.0 \end{minipage} \\ \tabularnewline

    \begin{minipage}[t]{0.18\columnwidth}\raggedright mariadb-java-client \end{minipage} & \begin{minipage}[t]{0.10\columnwidth}\raggedright 3.4.1 \end{minipage} & \begin{minipage}[t]{0.49\columnwidth}\raggedright Driver JDBC para base de datos MariaDB. \end{minipage} & \begin{minipage}[t]{0.11\columnwidth}\raggedright LGPL-2.1 \end{minipage} \\ \tabularnewline

    \begin{minipage}[t]{0.18\columnwidth}\raggedright spring-boot-starter-test \end{minipage} & \begin{minipage}[t]{0.10\columnwidth}\raggedright 3.4.3 \end{minipage} & \begin{minipage}[t]{0.49\columnwidth}\raggedright Starter para pruebas unitarias y de integración en Spring. \end{minipage} & \begin{minipage}[t]{0.11\columnwidth}\raggedright Apache-2.0 \end{minipage} \\ \tabularnewline

    \begin{minipage}[t]{0.18\columnwidth}\raggedright spring-security-test \end{minipage} & \begin{minipage}[t]{0.10\columnwidth}\raggedright 6.4.3 \end{minipage} & \begin{minipage}[t]{0.49\columnwidth}\raggedright Utilidades para pruebas de seguridad con Spring Security. \end{minipage} & \begin{minipage}[t]{0.11\columnwidth}\raggedright Apache-2.0 \end{minipage} \\ \tabularnewline

    \begin{minipage}[t]{0.18\columnwidth}\raggedright jjwt-api \end{minipage} & \begin{minipage}[t]{0.10\columnwidth}\raggedright 0.12.3 \end{minipage} & \begin{minipage}[t]{0.49\columnwidth}\raggedright API para creación y verificación de JSON Web Tokens (JWT). \end{minipage} & \begin{minipage}[t]{0.11\columnwidth}\raggedright Apache-2.0 \end{minipage} \\ \tabularnewline

    \begin{minipage}[t]{0.18\columnwidth}\raggedright jacoco \end{minipage} & \begin{minipage}[t]{0.10\columnwidth}\raggedright 0.8.12 \end{minipage} & \begin{minipage}[t]{0.49\columnwidth}\raggedright Agente para la recolección de cobertura de código en tiempo de ejecución. \end{minipage} & \begin{minipage}[t]{0.11\columnwidth}\raggedright EPL-2.0 \end{minipage} \\ \tabularnewline

    \begin{minipage}[t]{0.18\columnwidth}\raggedright junit-jupiter-api \end{minipage} & \begin{minipage}[t]{0.10\columnwidth}\raggedright 5.11.4 \end{minipage} & \begin{minipage}[t]{0.49\columnwidth}\raggedright API principal de JUnit 5 para pruebas unitarias. \end{minipage} & \begin{minipage}[t]{0.11\columnwidth}\raggedright EPL-2.0 \end{minipage} \\ \tabularnewline

    \begin{minipage}[t]{0.18\columnwidth}\raggedright junit-jupiter-engine \end{minipage} & \begin{minipage}[t]{0.10\columnwidth}\raggedright 5.11.4 \end{minipage} & \begin{minipage}[t]{0.49\columnwidth}\raggedright Motor de ejecución de pruebas para JUnit 5. \end{minipage} & \begin{minipage}[t]{0.11\columnwidth}\raggedright EPL-2.0 \end{minipage} \\ \tabularnewline
    \bottomrule
    \caption{Dependencias principales del proyecto Java}
\end{longtable}


\subsubsection{Angular}

\begin{longtable}[]{@{}p{0.25\columnwidth}p{0.10\columnwidth}p{0.49\columnwidth}p{0.11\columnwidth}@{}}
    \toprule
    \textbf{Dependencia} & \textbf{Versión} & \textbf{Descripción} & \textbf{Licencia} \\
    \midrule
    \endfirsthead

    \multicolumn{4}{c}%
    {\tablename\ \thetable{} -- \textit{continuación}} \\
    \toprule
    \textbf{Dependencia} & \textbf{Versión} & \textbf{Descripción} & \textbf{Licencia} \\
    \midrule
    \endhead

    \begin{minipage}[t]{0.18\columnwidth}\raggedright @angular/common \end{minipage} & \begin{minipage}[t]{0.10\columnwidth}\raggedright 19.2.0 \end{minipage} & \begin{minipage}[t]{0.49\columnwidth}\raggedright Funcionalidades comunes y directivas básicas de Angular. \end{minipage} & \begin{minipage}[t]{0.11\columnwidth}\raggedright MIT \end{minipage} \\ \tabularnewline

    \begin{minipage}[t]{0.18\columnwidth}\raggedright @angular/compiler \end{minipage} & \begin{minipage}[t]{0.10\columnwidth}\raggedright 19.2.0 \end{minipage} & \begin{minipage}[t]{0.49\columnwidth}\raggedright Compilador para templates de Angular. \end{minipage} & \begin{minipage}[t]{0.11\columnwidth}\raggedright MIT \end{minipage} \\ \tabularnewline

    \begin{minipage}[t]{0.18\columnwidth}\raggedright @angular/core \end{minipage} & \begin{minipage}[t]{0.10\columnwidth}\raggedright 19.2.0 \end{minipage} & \begin{minipage}[t]{0.49\columnwidth}\raggedright Núcleo del framework Angular. \end{minipage} & \begin{minipage}[t]{0.11\columnwidth}\raggedright MIT \end{minipage} \\ \tabularnewline

    \begin{minipage}[t]{0.18\columnwidth}\raggedright @angular/forms \end{minipage} & \begin{minipage}[t]{0.10\columnwidth}\raggedright 19.2.0 \end{minipage} & \begin{minipage}[t]{0.49\columnwidth}\raggedright Módulo para crear formularios reactivos y por plantilla. \end{minipage} & \begin{minipage}[t]{0.11\columnwidth}\raggedright MIT \end{minipage} \\ \tabularnewline

    \begin{minipage}[t]{0.18\columnwidth}\raggedright @angular/platform-browser \end{minipage} & \begin{minipage}[t]{0.10\columnwidth}\raggedright 19.2.0 \end{minipage} & \begin{minipage}[t]{0.49\columnwidth}\raggedright Soporte para la ejecución de Angular en navegadores. \end{minipage} & \begin{minipage}[t]{0.11\columnwidth}\raggedright MIT \end{minipage} \\ \tabularnewline

    \begin{minipage}[t]{0.18\columnwidth}\raggedright @angular/platform-browser-dynamic \end{minipage} & \begin{minipage}[t]{0.10\columnwidth}\raggedright 19.2.0 \end{minipage} & \begin{minipage}[t]{0.49\columnwidth}\raggedright Renderizado dinámico de Angular en el navegador. \end{minipage} & \begin{minipage}[t]{0.11\columnwidth}\raggedright MIT \end{minipage} \\ \tabularnewline

    \begin{minipage}[t]{0.18\columnwidth}\raggedright @angular/router \end{minipage} & \begin{minipage}[t]{0.10\columnwidth}\raggedright 19.2.0 \end{minipage} & \begin{minipage}[t]{0.49\columnwidth}\raggedright Sistema de enrutamiento y navegación para Angular. \end{minipage} & \begin{minipage}[t]{0.11\columnwidth}\raggedright MIT \end{minipage} \\ \tabularnewline

    \begin{minipage}[t]{0.18\columnwidth}\raggedright rxjs \end{minipage} & \begin{minipage}[t]{0.10\columnwidth}\raggedright 7.8.0 \end{minipage} & \begin{minipage}[t]{0.49\columnwidth}\raggedright Biblioteca para programación reactiva con observables. \end{minipage} & \begin{minipage}[t]{0.11\columnwidth}\raggedright Apache-2.0 \end{minipage} \\ \tabularnewline

    \begin{minipage}[t]{0.18\columnwidth}\raggedright tslib \end{minipage} & \begin{minipage}[t]{0.10\columnwidth}\raggedright 2.3.0 \end{minipage} & \begin{minipage}[t]{0.49\columnwidth}\raggedright Helpers de TypeScript para reducir el tamaño del bundle. \end{minipage} & \begin{minipage}[t]{0.11\columnwidth}\raggedright 0BSD \end{minipage} \\ \tabularnewline

    \begin{minipage}[t]{0.18\columnwidth}\raggedright zone.js \end{minipage} & \begin{minipage}[t]{0.10\columnwidth}\raggedright 0.15.0 \end{minipage} & \begin{minipage}[t]{0.49\columnwidth}\raggedright Implementación de zonas para la gestión del contexto de ejecución. \end{minipage} & \begin{minipage}[t]{0.11\columnwidth}\raggedright MIT \end{minipage} \\ \tabularnewline

    \begin{minipage}[t]{0.18\columnwidth}\raggedright @fortawesome \end{minipage} & \begin{minipage}[t]{0.10\columnwidth}\raggedright 6.7.2 \end{minipage} & \begin{minipage}[t]{0.49\columnwidth}\raggedright Biblioteca de iconos escalables basada en Font Awesome. \end{minipage} & \begin{minipage}[t]{0.11\columnwidth}\raggedright MIT \end{minipage} \\ \tabularnewline

    \begin{minipage}[t]{0.18\columnwidth}\raggedright tailwindcss \end{minipage} & \begin{minipage}[t]{0.10\columnwidth}\raggedright 4.1.4 \end{minipage} & \begin{minipage}[t]{0.49\columnwidth}\raggedright Framework de utilidad para crear interfaces modernas y responsivas. \end{minipage} & \begin{minipage}[t]{0.11\columnwidth}\raggedright MIT \end{minipage} \\ \tabularnewline

    \begin{minipage}[t]{0.18\columnwidth}\raggedright daisyui \end{minipage} & \begin{minipage}[t]{0.10\columnwidth}\raggedright 5.0.27 \end{minipage} & \begin{minipage}[t]{0.49\columnwidth}\raggedright Plugin de componentes UI para Tailwind CSS. \end{minipage} & \begin{minipage}[t]{0.11\columnwidth}\raggedright MIT \end{minipage} \\ \tabularnewline
    \bottomrule
    \caption{Dependencias principales del proyecto Angular}
\end{longtable}

Una vez estudiadas las dependencias y tipos de licencias de nuestro proyecto, observamos que las usadas son:

\begin{itemize}
    \item \textbf{MIT}
    \item \textbf{Apache-2.0}
    \item \textbf{BSD}
    \item \textbf{HPND}
    \item \textbf{0BSD}
\end{itemize}

Se ha optado por utilizar una licencia menos permisiva para no tener problemas con el uso de otros tipos de licencias.
En este caso elegimos Apache License 2.0\cite{apacheLicense}:

\begin{longtable}[]{@{}p{0.25\columnwidth}p{0.70\columnwidth}@{}}
    \toprule
    \textbf{Característica} & \textbf{Descripción} \\
    \midrule
    \endfirsthead

    \multicolumn{2}{c}{\tablename\ \thetable{} -- \textit{continuación}} \\
    \toprule
    \textbf{Característica} & \textbf{Descripción} \\
    \midrule
    \endhead

    \textbf{Libertad de uso} & Permite usar el software para cualquier propósito, incluyendo comercial. \\
    \textbf{Distribución} & Puedes redistribuir el código original o modificado, incluso bajo una licencia diferente compatible. \\
    \textbf{Modificación} & Puedes modificar el código fuente, pero debes indicar los cambios realizados. \\
    \textbf{Licencia permisiva} & No impone que el trabajo derivado use la misma licencia (no es copyleft). \\
    \textbf{Aviso de licencia} & Se debe incluir un aviso de derechos de autor y la licencia en todas las distribuciones. \\
    \textbf{Compatibilidad} & Compatible con muchas otras licencias, incluyendo GPL v3. \\
    \textbf{Patentes} & Concede una licencia explícita de uso de patentes del contribuyente; si alguien inicia una demanda de patentes, pierde la licencia. \\
    \textbf{Marcas registradas} & No concede ningún derecho sobre marcas registradas. \\
    \bottomrule
    \caption{Resumen de los puntos clave de la licencia Apache License 2.0}
\end{longtable}

De esta manera, aumentamos la posibilidad de que otros desarrolladores o empresas contribuyan a nuestro proyecto, pudiendo
adoptar el software más fácilmente.


\textbf{Documentación}

Con tal de proteger la autoría del proyecto, que sea fácil de aplicar y compatible con usos académicos y comercial,
se ha escogido Creative Commons Attribution 4.0 International\cite{ccby4}

\textbf{Imágenes y vídeos}

Para las imágenes y vídeos, utilizamos la licencia Creative Commons Attribution-NonCommercial 4.0 International\cite{ccby_nc4},
ya que las imágenes utilizadas son propias; así, protegemos el contenido pero permitimos su difusión.


\textbf{Resumen}

\begin{table}[H]
    \centering
    \begin{tabular}{|p{0.25\textwidth}|p{0.25\textwidth}|p{0.4\textwidth}|}
        \hline
        \textbf{Tipo de contenido} & \textbf{Licencia} & \textbf{Resumen de derechos y restricciones} \\
        \hline
        Código fuente & Apache License 2.0 & Permite uso, modificación, distribución (incluso comercial), siempre que se incluya un aviso de licencia y derechos de autor. No obliga a licenciar el código derivado bajo la misma licencia. \\
        \hline
        Documentación & CC BY 4.0 & Permite compartir y adaptar, incluso con fines comerciales, siempre que se dé atribución adecuada al autor original. \\
        \hline
        Imágenes y videos & CC BY-NC 4.0 & Permite compartir y adaptar, pero no para uso comercial. Se requiere atribución al autor original. \\
        \hline
    \end{tabular}
    \vspace{0.3em}
    \caption{Resumen de las licencias del proyecto.}
    \label{tab:licencias_contenido}
\end{table}
