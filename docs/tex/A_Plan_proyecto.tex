\apendice{Plan de Proyecto Software}

\section{Introducción}

La fase de planificación constituye un elemento clave e imprescindible en cualquier proyecto.
Mediante una planificación adecuada, se incrementan notablemente las posibilidades de éxito, ya
que permite anticipar problemas, distribuir adecuadamente los recursos y organizar las tareas de una
forma eficiente.

Por este motivo, en este anexo se detallan aquellos aspectos necesarios para el desarrollo
del proyecto.

Dividiremos el plan en dos apartados:
\begin{itemize}
    \item \textbf{La planificación temporal:} donde se establece el modo de organizar las actividades a realizar,
    una estimación de estas y las fechas de inicio y fin del proyecto.
    \item \textbf{El estudio de viabilidad:} en el que se analiza si el proyecto es factible desde el punto de vista
financiero y legal, atendiendo a los elementos necesarios para su realización.
\end{itemize}

\section{Planificación temporal}

El proyecto se ha realizado bajo un marco ágil de gestión de proyectos, basado en Scrum, aunque ha sido adaptado a las
características específicas del contexto.


Debido a la inexistencia de un equipo, la falta de un entorno estructurado (roles definidos, reuniones diarias, retrospectivas
y demos), no se ha aplicado en su totalidad.
Sin embargo, se han mantenido varios de sus fundamentos claves, con ciertos matices:

\begin{itemize}
    \item \textbf{Iteraciones cortas:} organizadas en sprints de 2 semanas, durante los cuales se han desarrollado
    nuevas funcionalidades de manera incremental.
    \item \textbf{Equipos autoorganizados:} aunque no exista un equipo como tal, el desarrollo se ha producido de una manera autónoma.
    \item \textbf{Reuniones semanales:} todos los lunes se programaba una reunión formal para revisar el estado del proyecto, y
    resolver posibles dudas surgidas durante la semana anterior.
    \item \textbf{Backlog del producto:} en el que se van añadiendo todas las tareas pendientes de realizar, ya sean de
    desarrollo, documentación\ldots Y se mantiene un listado de tareas ordenadas por prioridad.
    \item \textbf{Planificación de los sprints:} en el que se estudian las tareas que se realizarán para entregar en el
    siguiente incremento, atendiendo a características como prioridad y complejidad.
\end{itemize}

A cada tarea se le asigna un valor estimado de esfuerzo, denominado ``story points`` basados en la secuencia de
Fibonacci: 1, 2, 3, 5, 8\ldots Un story point representa una combinación de dificultad y tiempo estimado.
Es decir, una tarea con 3 puntos comparada con una de 5 es considerada más sencilla, y se realizará en menor tiempo.
No obstante, este sistema no establece una equivalencia exacta con días de trabajo.

Esta forma de trabajar no es la más exacta, no obstante permite flexibilizar el trabajo, lo que resulta útil en este contexto
específico en el que el tiempo disponible puede variar en el tiempo.
Incluso en entornos ideales, los equipos pueden verse afectados por situaciones como bajas, vacaciones\ldtos por eso la
planificación debe poder ser adaptable.


\textbf{Sprint 0 (25/02/2025 - 05/03/2025)}

Este primer sprint sirvió como punto de partida, se estableció una reunión inicial con el tutor, \textbf{Pedro Renedo}, en el que
tratamos de conocer acerca del proyecto que se pretende realizar.

Se creó el repositorio inicial en GitHub y estudiamos las posibles herramientas y lenguajes de programación que se utilizarían
en el desarrollo.

Se estimaron  \textbf{7 story points}, que aproximadamente se corresponde con la semana que duró el sprint.
Esta duración fue una excepción, ya que se trató del primer sprint, después se optó por dos semanas, porque una única
semana se hace muy corto cuando se introducen tareas de desarrollo.

\textbf{Sprint 1 (05/03/2025 - 20/03/2025)}

Durante este sprint se profundizó en el análisis del proyecto, tratando de responder a preguntas clave como:
\textbf{¿Qué se necesita realizar?, ¿Cómo ha de hacerse?, ¿Con base a qué surge esta necesidad? \ldots}

Además, se estudia el tipo de aplicación que se va a necesitar en los dispositivos, se modela una primera versión de
base de datos y se crea la estructura inicial de la aplicación backend en Java.
Surge un problema técnico, los ordenadores del centro son clonados, por lo que tienen el mismo GUID y no sirve como
identificador único, por lo que hay que seguir investigando para encontrar algún método alternativo.

Se estimaron \textbf{19 story points} de trabajo para los 15 días de trabajo.


\textbf{Sprint 2 (20/03/2025 - 02/04/2025)}

Se sigue con la línea de definición de requisitos y objetivos funcionales, se desarrollan los CU principales, además de
diseñar un método de autentificación para los equipos, que combina el GUID con el nombre de los equipos, aprovechando
que estos sí son únicos para cada uno.

Por otro lado, se crea la estructura de la aplicación frontal.

Se estimaron \textbf{15.5 story points} para los 15 días de trabajo, en este caso, se ajustó a 1 story point por día prácticamente.


\textbf{Sprint 3 (02/04/2025 - 16/04/2025)}

En esta etapa se comienza con las tareas de desarrollo del backend, para ello, se crea una base de datos inicial
con MariaDB, después de refinar el diagrama E-R actual, y se implementan las bases de nuestra API con Spring Boot, configurando
los primeros end points e implementando medidas de seguridad en las llamadas mediante login y generación de tokens.

Se estimaron \textbf{21 story points} en este sprint, ya que la disponibilidad prevista era superior a la de otros sprints.

\section{Estudio de viabilidad}

\subsection{Viabilidad económica}

En este apartado analizaremos los costes y beneficios de haber implementado el proyecto con una intención real de llevarlo
a cabo.

\textbf{Costes}

\textbf{Costes de personal:} todo el desarrollo ha sido realizado por un único desarrollador en un tiempo total de cinco meses, de febrero
a junio.
El sueldo base promedio para un puesto de desarrollador ``Full Stack Developer`` en España es de 30 mil €/año brutos de acuerdo con
la información publicada en Glassdor\cite{glassdoorSueldo}


Calculamos el salario con base a una persona de 30 años, soltera y sin hijos\cite{seguridadSocial2025}

\begin{center}
\begin{tabular}{| c | c |}\hline
\Xhline{2\arrayrulewidth}
\textbf{Concepto} & \textbf{Coste} \\ \Xhline{2\arrayrulewidth}
Salario neto & 1929,8€ \\ \hline
Retención IRPF & 411,5€ \\ \hline
Cuota seguridad social & 148,75€ \\ \hline
Salario mensual bruto & 2500€ \\ \hline
\Xhline{2\arrayrulewidth}
\textbf{Total 5 meses} & \textbf{12500€} \\ \Xhline{2\arrayrulewidth}
\end{tabular}


\vspace{0.3em}
\textit{Tabla: Costes de empleado.}
\end{center}

\textbf{Costes de hardware}

Para el hardware utilizado, no se emplean demasiadas herramientas más allá de un ordenador para la realización del proyecto,
que se hará a medida, a priori, con un dispositivo de 16GB de RAM, SSD NVMe, procesador actual y una tarjeta gráfica estándard
será suficiente.
En el precio viene ya incluido una licencia de Win 11.


\begin{center}
    \begin{tabular}{| c | c |}\hline
    \Xhline{2\arrayrulewidth}
    \textbf{Concepto} & \textbf{Coste} \\ \Xhline{2\arrayrulewidth}
    Ordenador a medida & 1200€ \\ \hline
    \Xhline{2\arrayrulewidth}
    \textbf{Total} & \textbf{1200€} \\ \Xhline{2\arrayrulewidth}
    \end{tabular}


    \vspace{0.3em}
    \textit{Tabla: Costes del hardware.}
\end{center}

\textbf{Costes de software}

En este apartado, tendremos en cuenta las licencias necesarias para el desarrollo del proyecto, para los IDE's, existen
dos opciones, o bien utilizar las licencias gratuitas ``Community Edition`` o las de pago\cite{jetbrainsPrecios}
Escogemos esta última porque cuentan con características que ayudan a desarrollar rápidamente nuestro proyecto.

\begin{center}
    \begin{tabular}{| c | c | c |}\hline
    \Xhline{2\arrayrulewidth}
    \textbf{Concepto} & \textbf{Coste} & \textbf{Coste Amortizado} \\ \Xhline{2\arrayrulewidth}
    All Products Pack & 474,8€ & 19,78€ \\ \hline
    Visual Studio Code & 0€ & 0€ \\ \hline
    \Xhline{2\arrayrulewidth}
    \textbf{Total} & \textbf{474,8€} & \textbf{19,78€} \\ \Xhline{2\arrayrulewidth}
    \end{tabular}


    \vspace{0.3em}
    \textit{Tabla: Costes del software.}
\end{center}

\textbf{Costes variados}

En este apartado incluimos otros costes a tener en cuenta

\begin{center}
    \begin{tabular}{| c | c |}\hline
    \Xhline{2\arrayrulewidth}
    \textbf{Concepto} & \textbf{Coste} \\ \Xhline{2\arrayrulewidth}
    Dominio web & 20€ \\ \hline
    Hosting / contendor & 50€\\ \hline
    Internet & 100€ \\ \hline
    \Xhline{2\arrayrulewidth}
    \textbf{Total} & \textbf{170€} \\ \Xhline{2\arrayrulewidth}
    \end{tabular}


    \vspace{0.3em}
    \textit{Tabla: Costes variados.}
\end{center}


\textbf{Costes totales}

Se muestra el sumatorio de todos los costes:

\begin{center}
    \begin{tabular}{| c | c |}\hline
    \Xhline{2\arrayrulewidth}
    \textbf{Concepto} & \textbf{Coste} \\ \Xhline{2\arrayrulewidth}
    Personal & 12500€ \\ \hline
    Hardware & 1200€\\ \hline
    Software & 478,8€ \\ \hline
    Variados & 170€ \\ \hline
    \Xhline{2\arrayrulewidth}
    \textbf{Total} & \textbf{14348,8€} \\ \Xhline{2\arrayrulewidth}
    \end{tabular}


    \vspace{0.3em}
    \textit{Tabla: Costes totales.}
\end{center}

\textbf{Beneficios}

Se trata de un proyecto con fines estrictamente académicos, de hecho, la idea es poder llegar a ser desarrollado para
la Universidad de Burgos.

No obstante, se podría considerar establecer algún tipo de suscripción (ya sea mensual o anual) u otras entidades educativas
o incluso a empresas privadas.

\begin{center}
    \begin{tabular}{| c | c | c | c | }\hline
    \Xhline{2\arrayrulewidth}
    \textbf{Tipo} & \textbf{Objetivo} & \textbf{Precio mensual} & \textbf{Precio anual} \\ \Xhline{2\arrayrulewidth}
    Básica & Centros educativos pequeños & 49,90€ & 499€ \\ \hline
    Profesional & Instituciones grandes & 89,90€ & 899€ \\ \hline
    Empresarial & Empresas privadas & 149,90€ & 1149€ \\ \hline
    Premium & Incluye personalización total & 250€ & 2500€ \\ \hline
    \end{tabular}


    \vspace{0.3em}
    \textit{Tabla: Monetización del proyecto.}
\end{center}

Para calcular el número de suscripciones que necesitaríamos para cubrir gastos, utilizamos la siguiente fórmula:

\[
    \text{Suscripciones necesarias} = \frac{14.348{,}8\,€}{\text{Precio de la suscripción anual}}
\]

\begin{center}
    \begin{tabular}{| c | c | c | }\hline
    \Xhline{2\arrayrulewidth}
    \textbf{Tipo} & \textbf{Precio anual} & \textbf{Suscripciones necesarias} \\ \Xhline{2\arrayrulewidth}
    Básica & 499€ & 29 \\ \hline
    Profesional & 899€ & 16 \\ \hline
    Empresarial & 1149€ & 13 \\ \hline
    Premium & 2500€ & 6 \\ \hline
    \end{tabular}


    \vspace{0.3em}
    \textit{Tabla: Suscripciones necesarias para cubrir gastos.}
\end{center}

De esta manera, calculamos el número de suscripciones necesarias (si solo se tuviesen de un mismo tipo)
para llegar a cubrir los gastos totales del proyecto.
También se podría utilizar un modelo mixto, e incluir suscripciones de cada tipo.

\textbf{Responsabilidad social}

Si este proyecto se implementase de manera real, uno de los aspectos más importantes a tener en cuenta es el impacto
sobre los trabajadores que actualmente desempeñan manualmente este tipo de tareas.
Es cierto que para este proyecto siempre se va a necesitar un mínimo de una persona, que ha de ejecutar la aplicación
en cada cliente, al menos para su instalación y llevar un mantenimiento por si se produjese algún tipo de error.
Sin embargo, se necesitarían muchas menos personas de las que actualmente se dedican a este tipo de tareas de una manera
más ``manual``, por lo que muchos puestos de trabajos actuales se verían afectados.
Por eso, se propone una formación continúa para el personal afectado, de manera que su perfil profesional pueda evolucionar
además de una trasparencia y comunicación interna que ayude a construir un entorno de confianza y adaptación responsable.

\subsection{Viabilidad legal}

Abordamos aquellos temas relacionados con el contexto legal del proyecto así como el de las licencias.

La implementación del proyecto ha sido propia, a excepción de:

\textbf{Software}

En este punto, diferenciaremos las dependencias utilizadas en el cliente (Python) de las utilizadas en la API (Java) y
la interfaz de gestión de usuario (Angular)

Python

\begin{longtable}[]{@{}llll@{}}
    \toprule
    \textbf{Dependencia} & \textbf{Versión} & \textbf{Descripción} & \textbf{Licencia} \\
    \midrule
    \endfirsthead

    \multicolumn{4}{c}%
    {\tablename\ \thetable{} -- \textit{continuación}} \\
    \toprule
    \textbf{Dependencia} & \textbf{Versión} & \textbf{Descripción} & \textbf{Licencia} \\
    \midrule
    \endhead

    \begin{minipage}[t]{0.18\columnwidth}\raggedright\strut altgraph \strut \end{minipage} & \begin{minipage}[t]{0.10\columnwidth}\raggedright\strut 0.17.4 \strut \end{minipage} & \begin{minipage}[t]{0.49\columnwidth}\raggedright\strut Análisis y representación de grafos de objetos de Python. \strut \end{minipage} & \begin{minipage}[t]{0.11\columnwidth}\raggedright\strut MIT \strut \end{minipage} \tabularnewline

    \begin{minipage}[t]{0.18\columnwidth}\raggedright\strut certifi \strut \end{minipage} & \begin{minipage}[t]{0.10\columnwidth}\raggedright\strut 2025.1.31 \strut \end{minipage} & \begin{minipage}[t]{0.49\columnwidth}\raggedright\strut Certificados raíz para validar la seguridad SSL. \strut \end{minipage} & \begin{minipage}[t]{0.11\columnwidth}\raggedright\strut MPL-2.0 \strut \end{minipage} \tabularnewline

    \begin{minipage}[t]{0.18\columnwidth}\raggedright\strut charset-normalizer \strut \end{minipage} & \begin{minipage}[t]{0.10\columnwidth}\raggedright\strut 3.4.1 \strut \end{minipage} & \begin{minipage}[t]{0.49\columnwidth}\raggedright\strut Detección de codificación de texto similar a chardet. \strut \end{minipage} & \begin{minipage}[t]{0.11\columnwidth}\raggedright\strut MIT \strut \end{minipage} \tabularnewline

    \begin{minipage}[t]{0.18\columnwidth}\raggedright\strut idna \strut \end{minipage} & \begin{minipage}[t]{0.10\columnwidth}\raggedright\strut 3.10 \strut \end{minipage} & \begin{minipage}[t]{0.49\columnwidth}\raggedright\strut Soporte de nombres de dominio internacionalizados (IDNA). \strut \end{minipage} & \begin{minipage}[t]{0.11\columnwidth}\raggedright\strut BSD \strut \end{minipage} \tabularnewline

    \begin{minipage}[t]{0.18\columnwidth}\raggedright\strut packaging \strut \end{minipage} & \begin{minipage}[t]{0.10\columnwidth}\raggedright\strut 24.2 \strut \end{minipage} & \begin{minipage}[t]{0.49\columnwidth}\raggedright\strut Utilidades para manejo de versiones y requerimientos. \strut \end{minipage} & \begin{minipage}[t]{0.11\columnwidth}\raggedright\strut Apache-2.0 \strut \end{minipage} \tabularnewline

    \begin{minipage}[t]{0.18\columnwidth}\raggedright\strut pefile \strut \end{minipage} & \begin{minipage}[t]{0.10\columnwidth}\raggedright\strut 2023.2.7 \strut \end{minipage} & \begin{minipage}[t]{0.49\columnwidth}\raggedright\strut Lectura y modificación de archivos PE (Portable Executable). \strut \end{minipage} & \begin{minipage}[t]{0.11\columnwidth}\raggedright\strut MIT \strut \end{minipage} \tabularnewline

    \begin{minipage}[t]{0.18\columnwidth}\raggedright\strut pillow \strut \end{minipage} & \begin{minipage}[t]{0.10\columnwidth}\raggedright\strut 11.1.0 \strut \end{minipage} & \begin{minipage}[t]{0.49\columnwidth}\raggedright\strut Biblioteca de procesamiento de imágenes (fork de PIL). \strut \end{minipage} & \begin{minipage}[t]{0.11\columnwidth}\raggedright\strut HPND \strut \end{minipage} \tabularnewline

    \begin{minipage}[t]{0.18\columnwidth}\raggedright\strut pip \strut \end{minipage} & \begin{minipage}[t]{0.10\columnwidth}\raggedright\strut 25.0.1 \strut \end{minipage} & \begin{minipage}[t]{0.49\columnwidth}\raggedright\strut Instalador oficial de paquetes Python. \strut \end{minipage} & \begin{minipage}[t]{0.11\columnwidth}\raggedright\strut MIT \strut \end{minipage} \tabularnewline

    \begin{minipage}[t]{0.18\columnwidth}\raggedright\strut pyinstaller \strut \end{minipage} & \begin{minipage}[t]{0.10\columnwidth}\raggedright\strut 6.12.0 \strut \end{minipage} & \begin{minipage}[t]{0.49\columnwidth}\raggedright\strut Empaquetado de apps Python en ejecutables. \strut \end{minipage} & \begin{minipage}[t]{0.11\columnwidth}\raggedright\strut GPL-2.0 \strut \end{minipage} \tabularnewline

    \begin{minipage}[t]{0.18\columnwidth}\raggedright\strut pyinstaller-hooks-contrib \strut \end{minipage} & \begin{minipage}[t]{0.10\columnwidth}\raggedright\strut 2025.2 \strut \end{minipage} & \begin{minipage}[t]{0.49\columnwidth}\raggedright\strut Hooks adicionales para PyInstaller. \strut \end{minipage} & \begin{minipage}[t]{0.11\columnwidth}\raggedright\strut GPL-2.0 \strut \end{minipage} \tabularnewline

    \begin{minipage}[t]{0.18\columnwidth}\raggedright\strut pystray \strut \end{minipage} & \begin{minipage}[t]{0.10\columnwidth}\raggedright\strut 0.19.5 \strut \end{minipage} & \begin{minipage}[t]{0.49\columnwidth}\raggedright\strut Gestión de iconos en la bandeja del sistema. \strut \end{minipage} & \begin{minipage}[t]{0.11\columnwidth}\raggedright\strut MIT \strut \end{minipage} \tabularnewline

    \begin{minipage}[t]{0.18\columnwidth}\raggedright\strut pywin32-ctypes \strut \end{minipage} & \begin{minipage}[t]{0.10\columnwidth}\raggedright\strut 0.2.3 \strut \end{minipage} & \begin{minipage}[t]{0.49\columnwidth}\raggedright\strut Acceso a funcionalidades Win32 usando ctypes. \strut \end{minipage} & \begin{minipage}[t]{0.11\columnwidth}\raggedright\strut MIT \strut \end{minipage} \tabularnewline

    \begin{minipage}[t]{0.18\columnwidth}\raggedright\strut requests \strut \end{minipage} & \begin{minipage}[t]{0.10\columnwidth}\raggedright\strut 2.32.3 \strut \end{minipage} & \begin{minipage}[t]{0.49\columnwidth}\raggedright\strut Biblioteca HTTP simple y elegante para Python. \strut \end{minipage} & \begin{minipage}[t]{0.11\columnwidth}\raggedright\strut Apache-2.0 \strut \end{minipage} \tabularnewline

    \begin{minipage}[t]{0.18\columnwidth}\raggedright\strut setuptools \strut \end{minipage} & \begin{minipage}[t]{0.10\columnwidth}\raggedright\strut 78.1.0 \strut \end{minipage} & \begin{minipage}[t]{0.49\columnwidth}\raggedright\strut Herramienta de empaquetado y distribución. \strut \end{minipage} & \begin{minipage}[t]{0.11\columnwidth}\raggedright\strut MIT \strut \end{minipage} \tabularnewline

    \begin{minipage}[t]{0.18\columnwidth}\raggedright\strut six \strut \end{minipage} & \begin{minipage}[t]{0.10\columnwidth}\raggedright\strut 1.17.0 \strut \end{minipage} & \begin{minipage}[t]{0.49\columnwidth}\raggedright\strut Compatibilidad entre Python 2 y 3. \strut \end{minipage} & \begin{minipage}[t]{0.11\columnwidth}\raggedright\strut MIT \strut \end{minipage} \tabularnewline

    \begin{minipage}[t]{0.18\columnwidth}\raggedright\strut urllib3 \strut \end{minipage} & \begin{minipage}[t]{0.10\columnwidth}\raggedright\strut 2.3.0 \strut \end{minipage} & \begin{minipage}[t]{0.49\columnwidth}\raggedright\strut Cliente HTTP con soporte para conexión persistente. \strut \end{minipage} & \begin{minipage}[t]{0.11\columnwidth}\raggedright\strut MIT \strut \end{minipage} \tabularnewline

    \bottomrule
    \caption{Dependencias del proyecto Python}
\end{longtable}


Java

\begin{longtable}[]{@{}p{0.18\columnwidth}p{0.10\columnwidth}p{0.49\columnwidth}p{0.11\columnwidth}@{}}
    \toprule
    \textbf{Dependencia} & \textbf{Versión} & \textbf{Descripción} & \textbf{Licencia} \\
    \midrule
    \endfirsthead

    \multicolumn{4}{c}%
    {\tablename\ \thetable{} -- \textit{continuación}} \\
    \toprule
    \textbf{Dependencia} & \textbf{Versión} & \textbf{Descripción} & \textbf{Licencia} \\
    \midrule
    \endhead

    \begin{minipage}[t]{0.18\columnwidth}\raggedright spring-boot-starter \end{minipage} & \begin{minipage}[t]{0.10\columnwidth}\raggedright 3.4.3 \end{minipage} & \begin{minipage}[t]{0.49\columnwidth}\raggedright Starter básico para aplicaciones Spring Boot. \end{minipage} & \begin{minipage}[t]{0.11\columnwidth}\raggedright Apache-2.0 \end{minipage} \\ \tabularnewline

    \begin{minipage}[t]{0.18\columnwidth}\raggedright spring-boot-starter-web \end{minipage} & \begin{minipage}[t]{0.10\columnwidth}\raggedright 3.4.3 \end{minipage} & \begin{minipage}[t]{0.49\columnwidth}\raggedright Starter para crear aplicaciones web con Spring MVC. \end{minipage} & \begin{minipage}[t]{0.11\columnwidth}\raggedright Apache-2.0 \end{minipage} \\ \tabularnewline

    \begin{minipage}[t]{0.18\columnwidth}\raggedright spring-boot-starter-actuator \end{minipage} & \begin{minipage}[t]{0.10\columnwidth}\raggedright 3.4.3 \end{minipage} & \begin{minipage}[t]{0.49\columnwidth}\raggedright Starter para monitorización y gestión de aplicaciones. \end{minipage} & \begin{minipage}[t]{0.11\columnwidth}\raggedright Apache-2.0 \end{minipage} \\ \tabularnewline

    \begin{minipage}[t]{0.18\columnwidth}\raggedright spring-boot-starter-data-jpa \end{minipage} & \begin{minipage}[t]{0.10\columnwidth}\raggedright 3.4.3 \end{minipage} & \begin{minipage}[t]{0.49\columnwidth}\raggedright Starter para usar Spring Data JPA con Hibernate. \end{minipage} & \begin{minipage}[t]{0.11\columnwidth}\raggedright Apache-2.0 \end{minipage} \\ \tabularnewline

    \begin{minipage}[t]{0.18\columnwidth}\raggedright spring-boot-starter-security \end{minipage} & \begin{minipage}[t]{0.10\columnwidth}\raggedright 3.4.3 \end{minipage} & \begin{minipage}[t]{0.49\columnwidth}\raggedright Starter para habilitar seguridad con Spring Security. \end{minipage} & \begin{minipage}[t]{0.11\columnwidth}\raggedright Apache-2.0 \end{minipage} \\ \tabularnewline

    \begin{minipage}[t]{0.18\columnwidth}\raggedright spring-boot-starter-validation \end{minipage} & \begin{minipage}[t]{0.10\columnwidth}\raggedright 3.4.3 \end{minipage} & \begin{minipage}[t]{0.49\columnwidth}\raggedright Starter para validación de datos usando Bean Validation. \end{minipage} & \begin{minipage}[t]{0.11\columnwidth}\raggedright Apache-2.0 \end{minipage} \\ \tabularnewline

    \begin{minipage}[t]{0.18\columnwidth}\raggedright mariadb-java-client \end{minipage} & \begin{minipage}[t]{0.10\columnwidth}\raggedright 3.4.1 \end{minipage} & \begin{minipage}[t]{0.49\columnwidth}\raggedright Driver JDBC para base de datos MariaDB. \end{minipage} & \begin{minipage}[t]{0.11\columnwidth}\raggedright LGPL-2.1 \end{minipage} \\ \tabularnewline

    \begin{minipage}[t]{0.18\columnwidth}\raggedright spring-boot-starter-test \end{minipage} & \begin{minipage}[t]{0.10\columnwidth}\raggedright 3.4.3 \end{minipage} & \begin{minipage}[t]{0.49\columnwidth}\raggedright Starter para pruebas unitarias y de integración en Spring. \end{minipage} & \begin{minipage}[t]{0.11\columnwidth}\raggedright Apache-2.0 \end{minipage} \\ \tabularnewline

    \begin{minipage}[t]{0.18\columnwidth}\raggedright spring-security-test \end{minipage} & \begin{minipage}[t]{0.10\columnwidth}\raggedright 6.4.3 \end{minipage} & \begin{minipage}[t]{0.49\columnwidth}\raggedright Utilidades para pruebas de seguridad con Spring Security. \end{minipage} & \begin{minipage}[t]{0.11\columnwidth}\raggedright Apache-2.0 \end{minipage} \\ \tabularnewline

    \begin{minipage}[t]{0.18\columnwidth}\raggedright jjwt-api \end{minipage} & \begin{minipage}[t]{0.10\columnwidth}\raggedright 0.12.3 \end{minipage} & \begin{minipage}[t]{0.49\columnwidth}\raggedright API para creación y verificación de JSON Web Tokens (JWT). \end{minipage} & \begin{minipage}[t]{0.11\columnwidth}\raggedright Apache-2.0 \end{minipage} \\ \tabularnewline

    \bottomrule
    \caption{Dependencias principales del proyecto Java}
\end{longtable}


Angular

\begin{longtable}[]{@{}p{0.25\columnwidth}p{0.10\columnwidth}p{0.49\columnwidth}p{0.11\columnwidth}@{}}
    \toprule
    \textbf{Dependencia} & \textbf{Versión} & \textbf{Descripción} & \textbf{Licencia} \\
    \midrule
    \endfirsthead

    \multicolumn{4}{c}%
    {\tablename\ \thetable{} -- \textit{continuación}} \\
    \toprule
    \textbf{Dependencia} & \textbf{Versión} & \textbf{Descripción} & \textbf{Licencia} \\
    \midrule
    \endhead

    \begin{minipage}[t]{0.18\columnwidth}\raggedright @angular/common \end{minipage} & \begin{minipage}[t]{0.10\columnwidth}\raggedright 19.2.0 \end{minipage} & \begin{minipage}[t]{0.49\columnwidth}\raggedright Funcionalidades comunes y directivas básicas de Angular. \end{minipage} & \begin{minipage}[t]{0.11\columnwidth}\raggedright MIT \end{minipage} \\ \tabularnewline

    \begin{minipage}[t]{0.18\columnwidth}\raggedright @angular/compiler \end{minipage} & \begin{minipage}[t]{0.10\columnwidth}\raggedright 19.2.0 \end{minipage} & \begin{minipage}[t]{0.49\columnwidth}\raggedright Compilador para templates de Angular. \end{minipage} & \begin{minipage}[t]{0.11\columnwidth}\raggedright MIT \end{minipage} \\ \tabularnewline

    \begin{minipage}[t]{0.18\columnwidth}\raggedright @angular/core \end{minipage} & \begin{minipage}[t]{0.10\columnwidth}\raggedright 19.2.0 \end{minipage} & \begin{minipage}[t]{0.49\columnwidth}\raggedright Núcleo del framework Angular. \end{minipage} & \begin{minipage}[t]{0.11\columnwidth}\raggedright MIT \end{minipage} \\ \tabularnewline

    \begin{minipage}[t]{0.18\columnwidth}\raggedright @angular/forms \end{minipage} & \begin{minipage}[t]{0.10\columnwidth}\raggedright 19.2.0 \end{minipage} & \begin{minipage}[t]{0.49\columnwidth}\raggedright Módulo para crear formularios reactivos y por plantilla. \end{minipage} & \begin{minipage}[t]{0.11\columnwidth}\raggedright MIT \end{minipage} \\ \tabularnewline

    \begin{minipage}[t]{0.18\columnwidth}\raggedright @angular/platform-browser \end{minipage} & \begin{minipage}[t]{0.10\columnwidth}\raggedright 19.2.0 \end{minipage} & \begin{minipage}[t]{0.49\columnwidth}\raggedright Soporte para la ejecución de Angular en navegadores. \end{minipage} & \begin{minipage}[t]{0.11\columnwidth}\raggedright MIT \end{minipage} \\ \tabularnewline

    \begin{minipage}[t]{0.18\columnwidth}\raggedright @angular/platform-browser-dynamic \end{minipage} & \begin{minipage}[t]{0.10\columnwidth}\raggedright 19.2.0 \end{minipage} & \begin{minipage}[t]{0.49\columnwidth}\raggedright Renderizado dinámico de Angular en el navegador. \end{minipage} & \begin{minipage}[t]{0.11\columnwidth}\raggedright MIT \end{minipage} \\ \tabularnewline

    \begin{minipage}[t]{0.18\columnwidth}\raggedright @angular/router \end{minipage} & \begin{minipage}[t]{0.10\columnwidth}\raggedright 19.2.0 \end{minipage} & \begin{minipage}[t]{0.49\columnwidth}\raggedright Sistema de enrutamiento y navegación para Angular. \end{minipage} & \begin{minipage}[t]{0.11\columnwidth}\raggedright MIT \end{minipage} \\ \tabularnewline

    \begin{minipage}[t]{0.18\columnwidth}\raggedright rxjs \end{minipage} & \begin{minipage}[t]{0.10\columnwidth}\raggedright 7.8.0 \end{minipage} & \begin{minipage}[t]{0.49\columnwidth}\raggedright Biblioteca para programación reactiva con observables. \end{minipage} & \begin{minipage}[t]{0.11\columnwidth}\raggedright Apache-2.0 \end{minipage} \\ \tabularnewline

    \begin{minipage}[t]{0.18\columnwidth}\raggedright tslib \end{minipage} & \begin{minipage}[t]{0.10\columnwidth}\raggedright 2.3.0 \end{minipage} & \begin{minipage}[t]{0.49\columnwidth}\raggedright Helpers de TypeScript para reducir el tamaño del bundle. \end{minipage} & \begin{minipage}[t]{0.11\columnwidth}\raggedright 0BSD \end{minipage} \\ \tabularnewline

    \begin{minipage}[t]{0.18\columnwidth}\raggedright zone.js \end{minipage} & \begin{minipage}[t]{0.10\columnwidth}\raggedright 0.15.0 \end{minipage} & \begin{minipage}[t]{0.49\columnwidth}\raggedright Implementación de zonas para la gestión del contexto de ejecución. \end{minipage} & \begin{minipage}[t]{0.11\columnwidth}\raggedright MIT \end{minipage} \\ \tabularnewline

    \bottomrule
    \caption{Dependencias principales del proyecto Angular}
\end{longtable}

Una vez estudiadas las dependencias y tipos de licencias de nuestro proyecto, observamos que las usadas son:

\begin{itemize}
    \item \textbf{MIT}
    \item \textbf{Apache-2.0}
    \item \textbf{BSD}
    \item \textbf{HPND}
    \item \textbf{0BSD}
\end{itemize}

Se ha optado por utilizar una licencia menos permisiva, para no tener problemas con el uso de otros tipos de licencias,
en este caso elegimos Apache License 2.0\cite{apacheLicense}:

\begin{longtable}[]{@{}p{0.25\columnwidth}p{0.70\columnwidth}@{}}
    \toprule
    \textbf{Característica} & \textbf{Descripción} \\
    \midrule
    \endfirsthead

    \multicolumn{2}{c}{\tablename\ \thetable{} -- \textit{continuación}} \\
    \toprule
    \textbf{Característica} & \textbf{Descripción} \\
    \midrule
    \endhead

    \textbf{Libertad de uso} & Permite usar el software para cualquier propósito, incluyendo comercial. \\
    \textbf{Distribución} & Puedes redistribuir el código original o modificado, incluso bajo una licencia diferente compatible. \\
    \textbf{Modificación} & Puedes modificar el código fuente, pero debes indicar los cambios realizados. \\
    \textbf{Licencia permisiva} & No impone que el trabajo derivado use la misma licencia (no es copyleft). \\
    \textbf{Aviso de licencia} & Se debe incluir un aviso de derechos de autor y la licencia en todas las distribuciones. \\
    \textbf{Compatibilidad} & Compatible con muchas otras licencias, incluyendo GPL v3. \\
    \textbf{Patentes} & Concede una licencia explícita de uso de patentes del contribuyente; si alguien inicia una demanda de patentes, pierde la licencia. \\
    \textbf{Marcas registradas} & No concede ningún derecho sobre marcas registradas. \\
    \bottomrule
    \caption{Resumen de los puntos clave de la licencia Apache License 2.0}
\end{longtable}

De esta manera, aumentamos la posibilidad de que otros desarrolladores o empresas contribuyan a nuestro proyecto, pudiendo
adoptar el software más fácilmente.


\textbf{Documentación}

Con tal de proteger la autoría del proyecto, que sea fácil de aplicar y compatible con usos académicos y comercial,
se ha escogido Creative Commons Attribution 4.0 International\cite{ccby4}

\textbf{Imágenes y vídeos}

Para las imágenes y vídeos, utilizamos la licencia Creative Commons Attribution-NonCommercial 4.0 International\cite{ccby_nc4},
ya que las imágenes utilizadas son propias, protegemos el contenido, pero permitimos su difusión.


\textbf{Resumen}

\begin{table}[H]
    \centering
    \begin{tabular}{|p{0.25\textwidth}|p{0.25\textwidth}|p{0.4\textwidth}|}
        \hline
        \textbf{Tipo de contenido} & \textbf{Licencia} & \textbf{Resumen de derechos y restricciones} \\
        \hline
        Código fuente & Apache License 2.0 & Permite uso, modificación, distribución (incluso comercial), siempre que se incluya un aviso de licencia y derechos de autor. No obliga a licenciar el código derivado bajo la misma licencia. \\
        \hline
        Documentación & CC BY 4.0 & Permite compartir y adaptar, incluso con fines comerciales, siempre que se dé atribución adecuada al autor original. \\
        \hline
        Imágenes y videos & CC BY-NC 4.0 & Permite compartir y adaptar, pero no para uso comercial. Se requiere atribución al autor original. \\
        \hline
    \end{tabular}
    \vspace{0.3em}
    \caption{Resumen de las licencias del proyecto.}
    \label{tab:licencias_contenido}
\end{table}
