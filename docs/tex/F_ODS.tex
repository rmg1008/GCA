\apendice{Anexo de sostenibilización curricular}

\section{Introducción}

La implementación del proyecto podría tener un impacto social relevante al reducir la necesidad de personal en
tareas manuales, lo que afectaría a ciertos puestos de trabajo.
Por ello, se propone acompañar esta transición con formación continua y una comunicación interna transparente que
facilite la adaptación del personal.

A nivel ambiental, se ha prestado especial atención al consumo energético del sistema, adoptando medidas como la
reducción de consultas innecesarias a la API, la optimización de recursos y el uso de formatos de datos ligeros, con
el fin de minimizar su impacto y garantizar una implementación más eficiente y sostenible.

\section{Sostenibilidad social}

Uno de los puntos clave a considerar en la sostenibilidad social del proyecto es el impacto que tendrá en el empleo,
aunque se requiere de al menos un operario para la supervisión del sistema, se prevé que la automatización de tareas
manuales afecte a ciertos puestos de trabajo.
Es recomendable que esta transición vaya acompañada de un plan de formación continua para el personal, permitiendo una
evolución hacia roles más especializados y técnicos.

\section{Sostenibilidad ambiental}

Se ha tenido especial atención al consumo energético del sistema, adoptando una serie de medidas para minimizar su
impacto ambiental.
Entre estas medidas se incluyen:
\begin{itemize}
    \item Reducción de consultas innecesarias a la API, optimizando el uso de recursos y evitando sobrecargas en el
    sistema.
    \item Uso de formatos de datos ligeros para minimizar el tamaño de las transferencias y reducir el consumo de
    ancho de banda.
    \item Implementación de prácticas de programación eficientes que reduzcan los tiempos de ejecución.
\end{itemize}

De cara a futuras implementaciones, se recomienda seguir investigando y aplicando tecnologías que permitan una mayor
eficiencia energética, además de monitorizar el consumo energético de los dispositivos y servicios implicados en el
proyecto.

\section{Conclusiones}

El proyecto no solo busca mejorar la eficiencia operativa, sino que también se compromete a un desarrollo más sostenible
e inclusivo, en línea con las directrices propuestas por la CRUE\cite{crueSostenibilidad} para formar al personal
de manera que se comprometan con su entorno y estén capacitados para actuar de manera responsable ante los retos
sociales y tecnológicos actuales.