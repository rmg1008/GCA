\apendice{Anexo de sostenibilización curricular}

\section{Introducción}
La implementación del proyecto podría tener un impacto social relevante al reducir la necesidad de personal en
tareas manuales, lo que afectaría a ciertos puestos de trabajo.
Por ello, se propone acompañar esta transición con formación continua y una comunicación interna transparente que
facilite la adaptación del personal.

A nivel ambiental, se ha prestado especial atención al consumo energético del sistema, adoptando medidas como la
reducción de consultas innecesarias a la API, la optimización de recursos y el uso de formatos de datos ligeros, con
el fin de minimizar su impacto y garantizar una implementación más eficiente y sostenible.
