\apendice{Documentación técnica de programación}

\section{Introducción}

El objetivo de este apéndice es ofrecer una visión clara y estructurada de los procesos de implementación de la
aplicación, de manera que pueda utilizarse por otros programadores como referencia para el mantenimiento y
el desarrollo de nuevas funcionalidades.
Al tratarse de documentación técnica, se entiende que el lector posee de nociones en la materia.


\section{Estructura de directorios}

El directorio del proyecto se puede consultar en el repositorio \href{https://github.com/rmg1008/GCA}{GitHub}.

La estructura de directorios del proyecto es la siguiente:
\begin{itemize}
    \tightlist
    \item\texttt{GCA/}: Directorio raíz del proyecto.
    Contiene todos los archivos y subdirectorios del proyecto,
    README.md, .env, docker-compose.yml y licencia.
    \item \texttt{GCA/.github/}: Acciones de GitHub.
    Contiene los archivos de configuración para las acciones de GitHub,
    utilizando workflows para la integración y despliegue continuo (CI/CD).
    \item\texttt{GCA/backend/}: Proyecto Spring Boot.
    Contiene el código fuente del backend, archivos de configuración y dependencias.
    \item \texttt{GCA/backend/src/main/java/}: Código fuente del backend.
    Contiene el código Java de la aplicación.
    \item \texttt{GCA/backend/src/main/resources/}: Recursos del backend.
    Contiene los archivos de configuración y ficheros para la base de datos.
    \item \texttt{GCA/backend/src/test/java}: Tests del backend.
    Contiene las pruebas unitarias y de integración.
    \item \texttt{GCA/backend/src/test/resources}: Recursos de pruebas.
    Contiene los archivos de configuración y ficheros necesarios para las pruebas.
    \item \texttt{GCA/client/}: Aplicación de escritorio Python.
    Contiene el código fuente de la aplicación de escritorio.
    \item \texttt{GCA/client/libs/}: Librerías del cliente.
    Contiene las librerías necesarias para la aplicación.
    \item \texttt{GCA/client/output/}: Instalador del cliente.
    \item \texttt{GCA/client/resources/}: Archivos de recursos del cliente.
    Contiene los archivos de configuración e imágenes.
    \item \texttt{GCA/docs/}: Documentación del proyecto.
    Contiene los documentos de diseño, manuales y otros archivos.
    \item \texttt{GCA/docs/img/}: Imágenes.
    \item \texttt{GCA/frontend/}: Proyecto Angular.
    Contiene el código fuente del frontend, archivos de configuración\ldots
    \item \texttt{GCA/frontend/public/}: Archivos públicos del frontend (imágenes, iconos\ldots).
    \item \texttt{GCA/frontend/src/}: Código fuente del frontend.
    Contiene el código TypeScript de la aplicación.
    \item \texttt{GCA/frontend/src/app/}: Código de la aplicación (componentes, servicios, modelos\ldots).
    \item \texttt{GCA/frontend/src/environments/}: Configuración de los distintos entornos.
\end{itemize}


\section{Manual del programador}

Como hemos mencionado en la introducción, este apéndice está destinado a programadores que deseen comprender la
implementación del proyecto.
Se detallan todos aquellos aspectos necesarios para la instalación, compilación y ejecución del proyecto, así como
las pruebas del sistema.

\section{Compilación, instalación y ejecución del proyecto}

En primer lugar, es importante mencionar que el proyecto se ha realizado bajo el sistema operativo Windows 11, por lo
que las instrucciones de instalación y ejecución están orientadas a este sistema operativo.
No obstante, la gran mayoría de instrucciones son aplicables a otros entornos.

El proyecto se divide en tres partes principales: el backend, el frontend y el cliente.
Aunque estén separadas, se encuentran en el mismo repositorio para favorecer una mayor facilidad de uso y despliegue.

Existen dos formas de ejecutar el proyecto: mediante Docker o de forma manual.

\subsection{Docker}

Para ejecutar el proyecto mediante Docker, es necesario tener instalado Docker en el sistema: se puede descargar
\href{https://www.docker.com/}{aquí}.

Al instalar Docker Desktop, se nos permite tener una interfaz gráfica para gestionar los contenedores, sin perjuicio de poder utilizarse
la línea de comandos si se prefiere.

Tras instalar Docker, podemos comprobar que se ha instalado correctamente desde nuestra terminal ejecutando:
\begin{verbatim}
docker --version
\end{verbatim}

Si se ha instalado correctamente, nos mostrará la versión de Docker instalada.
Iniciamos la aplicación Docker Desktop y nos aseguramos de que está en ejecución.

Nos dirigimos al directorio raíz del proyecto y ejecutamos el siguiente comando:
\begin{verbatim}
docker-compose up --build
\end{verbatim}

Si al ejecutarlo se produce un error del tipo \texttt{yaml: did not find expected key}, es posible que haya un error de
propiedades mal formadas en el archivo \texttt{docker-compose.yml}.

Si todo ha ido bien, se iniciarán los contenedores del backend, frontend y base de datos; podremos comprobarlo desde la
interfaz de Docker Desktop:

\imagen{docker.png}{Contenedores de Docker.}{0.5}

Una vez que los contenedores estén en ejecución, podemos acceder al frontend desde el navegador web en la dirección
\texttt{http://localhost:4200} y al backend en \texttt{http://localhost:8080} o las urls que hayamos configurado.

Para que funcione correctamente, la base de datos debe levantarse cuando se inicie el backend, ya que este utiliza
la base de datos para iniciar el contexto de la aplicación.
Esto se consigue mediante la configuración del docker-compose.yml, donde se especifica que el backend depende del
servicio de la base de datos.

\subsection{Manual}

\subsubsection{Backend}

Para ejecutar el backend de forma manual, es necesario tener instalar:
\begin{itemize}
    \tightlist
    \item JDK 17.
    \item Gradle.
    \item Git.
    \item MariaDB\@.
\end{itemize}

En este caso decidí utilizar IntelliJ IDEA como IDE, pero se puede utilizar cualquier otro IDE compatible con Java.

El proyecto se puede importar directamente utilizando la URL del repositorio de GitHub o clonando el repositorio y
después importándolo en el IDE\@.

Para este proyecto en concreto, es conveniente clonar el repositorio y abrirlo desde el IDE, ya que la carpeta del
proyecto que vamos a utilizar es la carpeta \texttt{backend/} y no la raíz del proyecto.

Para clonar el repositorio, abrimos una terminal y ejecutamos el siguiente comando:
\begin{verbatim}
git clone https://github.com/rmg1008/GCA.git
\end{verbatim}

Una vez clonado el repositorio, abrimos IntelliJ IDEA y seleccionamos la opción de abrir un proyecto existente:
seleccionamos la carpeta \texttt{GCA/backend/} e IntelliJ IDEA nos preguntará si queremos importar el proyecto como un
proyecto de Gradle; le decimos que sí.

\imagen{intellij.png}{Importar proyecto con Gradle.}{1}

Una vez importado el proyecto, IntelliJ IDEA descargará las dependencias necesarias y configurará el proyecto.
Es importante configurar en el proyecto la versión sdk de Java 17, pues es la versión sobre la que se ha desarrollado el
proyecto y es necesaria para que SonarQube funcione correctamente, porque no es compatible con versiones superiores.
Para ello, vamos a \texttt{Archivo > Estructura del proyecto > SDK del proyecto} y seleccionamos la versión 17.

\imagen{sdk.png}{Configuración del SDK de proyecto.}{1}

Se puede descargar a través del siguiente \href{https://www.azul.com/downloads/?version=java-17-lts&os=windows&package=jdk#zulu}{enlace}.

Una vez configurado el SDK, podemos proceder a configurar la base de datos.
Para ello, debemos crear una base de datos en MariaDB para el entorno de desarrollo.
Se puede descargar MariaDB desde su página oficial \href{https://mariadb.org/download}{aquí}.

No importa cómo se haya nombrado la base de datos, pero es importante que se configure correctamente en el archivo
\texttt{application.properties} que se encuentra en la carpeta \texttt{src/main/resources/} del proyecto.
En este archivo encontramos todas aquellas configuraciones que podemos modificar antes de la ejecución del proyecto.

IntelliJ nos ofrece la posibilidad de configurar la conexión a la base de datos desde el IDE: para ello, vamos a
\texttt{Ver > Herramientas de base de datos} y creamos una nueva conexión a la base de datos MariaDB. De esta forma,
podremos ver las tablas y datos de la base de datos sin salir del IDE.

Por último, para iniciar el proyecto, podemos utilizar la opción de lanzarlo desde IntelliJ IDEA, o bien
ejecutar el siguiente comando en la terminal:
\begin{verbatim}
./gradlew bootRun
\end{verbatim}

Si todo ha ido bien, el backend se iniciará y podremos acceder a la API REST desde la dirección \texttt{http://localhost:8080}.

\imagen{spring}{Ejecución del backend.}{1}

No obstante, es recomendable ejecutar el proyecto desde IntelliJ IDEA, ya que nos ofrece una mejor experiencia de
desarrollo y nos permite depurar el código ejecutándolo en modo debug.

\imagen{spring_start.png}{Ejecución del backend desde IntelliJ IDEA.}{0.5}

Para las llamadas a la API REST, hemos utilizado la herramienta Postman, que permite realizar peticiones HTTP
de forma sencilla y visual.
Se puede descargar desde su página oficial \href{https://www.postman.com/downloads/}{aquí}.

Todas las librerias y dependencias del proyecto se gestionan mediante Gradle.
Si deseamos añadir o modificar alguna extra, podemos hacerlo en el archivo \texttt{build.gradle} que se encuentra
en la carpeta raíz del proyecto.

\subsubsection{Frontend}

Para ejecutar el frontend de forma manual, es necesario tener instalado:

\begin{itemize}
    \tightlist
    \item Node.js.
    \item Angular CLI\@.
    \item Visual Studio Code\@.
\end{itemize}

NPM se instala junto a Node.js, por lo que no es necesario instalarlo de forma independiente.
Se puede descargar desde su página oficial \href{https://nodejs.org/es/download/}{aquí}.

Para instalar Angular CLI, podemos utilizar el siguiente comando en la terminal:
\begin{verbatim}
npm install -g @angular/cli
\end{verbatim}

Las versiones actuales de Angular CLI y Node.js son las siguientes:

\imagen{ng.png}{Versiones del proyecto frontend.}{1}

Una vez instalado, ejecutamos Visual Studio Code y abrimos la carpeta del proyecto \texttt{frontend/}.
Para una mejor experiencia de desarrollo, es recomendable instalar las extensiones de Angular y TypeScript.

Además, es necesario instalar las dependencias del proyecto la primera vez que lo importamos.
Para ello, abrimos una terminal en Visual Studio Code y ejecutamos el siguiente comando:
\begin{verbatim}
npm install
\end{verbatim}

De cara al futuro, si se desea añadir o modificar alguna dependencia, podemos hacerlo en el archivo
\texttt{package.json} que se encuentra en la carpeta raíz del proyecto, y después, ejecutar de nuevo el comando anterior
o bien utilizar NPM para instalar una dependencia concreta:
\begin{verbatim}
npm install nombre-dependencia
\end{verbatim}

Para iniciar el proyecto, abrimos una terminal en Visual Studio Code y ejecutamos el siguiente comando:
\begin{verbatim}
ng serve
\end{verbatim}

Una vez listo, el proyecto se iniciará y podremos acceder al frontend desde el navegador web en la dirección
\texttt{http://localhost:4200/dashboard}.

\subsubsection{Cliente}

Para importar el cliente de forma manual, es necesario tener instalado:
\begin{itemize}
    \tightlist
    \item Python 3.10 o superior.
    \item Pip (incluido con Python).
    \item PyInstaller.
    \item PyCharm o cualquier otro IDE compatible con Python.
    \item Inno Setup.
\end{itemize}

Se puede descargar Python para Windows desde su página oficial \href{https://www.python.org/downloads/windows/}{aquí}.

Para instalar PyInstaller, podemos utilizar el siguiente comando en la terminal:
\begin{verbatim}
pip install pyinstaller
\end{verbatim}

Para instalar PyCharm, podemos descargarlo desde su página oficial \href{https://www.jetbrains.com/pycharm/download/}{aquí}.

Para instalar Inno Setup, podemos descargarlo desde su página oficial \href{https://jrsoftware.org/isinfo.php}{aquí}.

Con todo instalado, abrimos PyCharm e importamos el proyecto desde la carpeta \texttt{client/}.

Desde PyCharm, podemos ejecutar el cliente directamente y utilizar el debug, siempre y cuando hayamos ejecutado el IDE
como administrador para que pueda acceder a los recursos del sistema y el entorno sea Windows.

Antes de crear el ejecutable, podemos modificar el archivo \texttt{config.json} que se encuentra en la carpeta
\texttt{resources/} del cliente y configurar la URL del backend, entre otras opciones.

Para crear un instalador del cliente, podemos ejecutar el siguiente comando en la terminal desde la carpeta del cliente:
\begin{verbatim}
 pyinstaller --uac-admin --noconsole --onefile --add-data
"resources/config.json;." --icon=resources/icono.ico main.py
\end{verbatim}
o bien ejecutar este otro comando:
\begin{verbatim}
     pyinstaller.exe .\main.spec
\end{verbatim}

Esto generará un archivo (.exe) en la carpeta \texttt{dist/} del cliente, que se puede ejecutar directamente.
Sin embargo, como queremos que el cliente pueda instalar en el sistema, utilizamos Inno Setup para crear un instalador.

InnoSetup funciona mediante scripts.
Tenemos ya un script preparado en la carpeta \texttt{client/} que podemos utilizar.
Abrimos Inno Setup y cargamos el script \texttt{gca\_script.iss} que se encuentra en la carpeta \texttt{client/}.

\imagen{inno.png}{Script de Inno Setup.}{1}

En el script se especifican las configuraciones del instalador, como el nombre del programa, la versión,
el icono, los archivos a incluir, etc.
El primer paso es compilar el script.
Para ello, pulsamos en el botón de compilar en la barra de herramientas, o bien utilizamos el atajo de teclado \texttt{Ctrl + F9}.

Una vez compilado, se generará un instalador en la carpeta \texttt{output/} del cliente con el nombre \texttt{GCA\_Installer.exe}.

Se pueden instalar nuevas dependencias en el cliente mediante PIP, al igual que en cualquier otro proyecto de Python.
No obstante, cabe destacar que cuantas más dependencias se añadan, peor compatibilidad podrá tener el instalador con
los dispositivos, pues algunas dependencias pueden requerir de librerías adicionales que no estén presentes en
el sistema.

\section{Pruebas del sistema}

Las pruebas se han realizado en el backend utilizando Junit y Spring Boot Test.
Todo el código de las pruebas se encuentra en la carpeta \texttt{src/test/java/} del proyecto y su configuración en
la carpeta \texttt{resources/}.

Se utiliza una base de datos en memoria H2, por lo que no es necesario tener una base de datos
configurada.
De esta forma, las pruebas son más rápidas y no afectan a la base de datos real.

Dentro de los tests se pueden encontrar pruebas unitarias y de integración: las pruebas unitarias no necesitan levantar
el contexto de la aplicación, mientras que las pruebas de integración sí lo necesitan, porque prueban la interacción entre
los distintos componentes de la aplicación.

En relación a esto, se ha añadido Jacoco para la cobertura de código, que permite ver el porcentaje de código cubierto
por las pruebas.

Se pueden ejecutar los tests de manera individual desde el IDE: basta con abrir las clases de tests y pulsar en el botón
de ejecutar:

\imagen{test.png}{Ejecución individual desde el IDE.}{0.5}

Si todo ha ido bien, aparece un tick verde en todos los métodos de prueba; en caso contrario, se marcan aquellos
test que han fallado con una cruz roja.

Otra opción es utilizar Gradle para ejecutar todos los tests del proyecto.
Para ello, abrimos una terminal en la carpeta \texttt{backend/} y ejecutamos el siguiente comando:
\begin{verbatim}
./gradlew test
\end{verbatim}

O bien, utilizamos la herramienta de IntelliJ que tiene integrada para ejecutar tas tareas Gradle.
Para ello, vamos a la barra lateral derecha y buscamos la pestaña de Gradle, y allí buscamos la tarea \texttt{test} y la ejecutamos:

\imagen{gradle.png}{Ejecución de tests desde Gradle.}{1}

Esta tarea ejecuta todos los tests del proyecto y genera un informe de los resultados en la carpeta
\texttt{build/jacocoHtml}.

Si todo ha ido bien, en Spring se muestra el resultado:

\imagen{test_results.png}{Resultado de pruebas.}{1}

Y podemos consultar el informe en nuestro navegador accediendo a la dirección
\texttt{../backend/build/jacocoHtml/index.html}.

\imagen{test_report.png}{Informe de pruebas.}{1}

En verde se muestran las líneas de código que han sido cubiertas por las pruebas y en rojo las que no,
además del porcentaje de cobertura total del proyecto.
Podemos navegar por el informe para ver en detalle la cobertura en cada clase y método (es decir, tanto las líneas de
código que han sido cubiertas por las pruebas como las que no).

\imagen{jacoco_example.png}{Ejemplo de detalle de cobertura de una clase.}{1}

\imagen{jacoco_example_2.png}{Ejemplo de detalle de cobertura de un método.}{1}

Además de todo esto, se ha utilizado SonarQube para el análisis del código.
Esta herramienta se ejecuta en la nube y es válida para analizar el código que se encuentra en la rama principal del
repositorio, dado que la versión utilizada es gratuita y no permite analizar ramas diferentes a la principal.

Los resultados del análisis se pueden consultar en la página de SonarQube del proyecto y muestra la siguiente información:
\imagen{sonarqube.png}{Análisis de SonarQube.}{1}

Para no tener que subir constantemente el proyecto a SonarQube, se ha instalado el siguiente plugin en IntelliJ IDEA:
\imagen{sonarqube_plugin.png}{Plugin de SonarQube para IntelliJ IDEA.}{1}

Esta herramienta nos permite analizar el código directamente desde el IDE y ver los resultados en tiempo real, como
el que muestra la siguiente imagen:
\imagen{sonarqube_ide.png}{Análisis de SonarQube desde el IDE.}{1}

Así, antes de subir el código al repositorio, podemos asegurarnos de que cumple con los estándares de calidad necesarios.

Por último, se ha incluido en GitHub Actions un workflow que se ejecuta cada vez que se realiza un push al repositorio;
se encarga de compilar el proyecto y ejecutar los tests, y puede comprobarse directamente desde GitHub:

\imagen{github_check.png}{GitHub Check.}{1}

Y pulsando sobre el check, podemos ver los detalles de la ejecución:

\imagen{github_check_details.png}{Detalles del GitHub Check.}{1}

De esta manera, se consigue que cada vez que un usuario realice una aportación al repositorio, se garantice que el código
cumple con los estándares de calidad y que no se introducen errores en el proyecto.